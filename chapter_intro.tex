\chapter{Introduction}
\section{Background}
\subsection{Why bioimage informatics matters}
\subsection{From \emph{in-vitro} to \emph{in-vivo} cell imaging}
\subsection{The cells in Zebra fish brain}
\section{Motivation and main contributions}
The movement of microglia cells and the growing of neuron cells.
\section{Thesis organization}
The rest of the thesis is organized as follows. Part 2 describes the cell tracking methods (chapter \ref{chpt:coseg}) and the related image segmentation (chapter \ref{chpt:imgseg}), component tree construction (chapter \ref{chpt:cptree}) and tree assignment methods (chapter \ref{chpt:treeassign}). Part 3 describes the automatic neuron tracing methods (chapter \ref{chpt:auto-nt}), human guided neuron tracing methods (chapter \ref{chpt:manual-nt}) and the powerful fastmarching method (chapter \ref{chpt:fm}).

