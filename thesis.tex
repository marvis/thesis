\documentclass[b5,12pt]{report}
\usepackage{graphicx}
\usepackage{multirow} % used in multi-row table
\usepackage{color} % set color for text
\usepackage[margin=1in]{geometry}  % set margin size
\usepackage[linesnumbered,boxed]{algorithm2e}  % display algorithm
\usepackage{amsmath}
\usepackage{amssymb}
\usepackage{hyperref} % add link for tableofcontents
\linespread{1.5}
\usepackage{anysize}
\usepackage{rotating}
\usepackage{threeparttable} % for tablenotes

\marginsize{3cm}{2cm}{1.2cm}{1.2cm} % need package anysize
\setcounter{secnumdepth}{3} % set the section depth
\setcounter{tocdepth}{2}
\title{Tracking the Movement of Microglia Cells and the Growing of Neurons in Zebra Fish Brain}
\begin{document}
\maketitle
\begin{abstract}
 This is cool paper about tracking.
\end{abstract}
\renewcommand{\abstractname}{Acknowledgements}
\begin{abstract}
 Thanks Mum!
\end{abstract}

\tableofcontents
\listoffigures
\part{Thesis overview}
\chapter{Introduction}
\section{Background}
\subsection{Why bioimage informatics matters}
\subsection{Current imaging techniques}
\subsection{The cells in Zebra fish brain}
\section{Motivation and my work}
The movement of microglia cells and the growing of neuron cells.
\section{Thesis organization}
The rest of the thesis is organized as follows. Part 2 describes the cell tracking methods (chapter \ref{chpt:coseg}) and the related image segmentation (chapter \ref{chpt:imgseg}), component tree construction (chapter \ref{chpt:cptree}) and tree assignment methods (chapter \ref{chpt:treeassign}). Part 3 describes the automatic neuron tracing methods (chapter \ref{chpt:auto-nt}), human guided neuron tracing methods (chapter \ref{chpt:manual-nt}) and the powerful fastmarching method (chapter \ref{chpt:fm}).
\part{Cell Tracking}
\chapter{Image segmentation} \label{chpt:imgseg}
For the cell tracking methods which seperate the segmentation and association steps, the segmentation quality influences the whole tracking result significantly. A good segmentation for the objects in each frame is very important especially in the very early frames. Because the segmentation results of priori frame affects the segmentation results of later frames, that the missing-segmentation of an object will be very likely to be miss tracked in later frames.

Image segmentation are basically ad-hoc problems. Different segmentation emphasize one or more of the desired properites. During the past 40 years, hundreds of segmentation algorithms have been proposed\cite{freixenet2002yet}. Some basic segmentation methods, such as global/local thresholding and some complex segmentation models, such as active contour, chanvese's method , and graph cut method, will be introduced in this chaper.

In this thesis, I proposed a voronoi-diagram based multi-cell segmentation method, which needs only one level set function. It is an extension of Dufour \cite{dufour2005segmenting} and Zhang's method \cite{zhang2004tracking} without considering the coupling constraints (usually unnecessary). The key idea of my method is to build a voronoi diagram for all cells by using fastmarching method. Then each object performs the chanvese's segmentation method in its own voronoi area. I build an image processing platform with lots of image segmentation algorithms in C/C++, which is distributed as an open source repository.
\section{Thresholding method}
\subsection{Overview}
In digital image processing, thresholding is the mostly used technique for image segmentation due to its easy usage. Normally the thresholding value is a single value which partitions a grayscale image into foreground and background area. It is offen an effective tool to separate objects from the background and it is always the first tried method before applying other complex segmentation methods. One application of thresholding is document image analysis which aims to extract printed characters \cite{kamel1993extraction,abak1997performance}, graphs, or other items. Examples of thresholding applications lies in all kinds of pre-processing or post-processing steps, including edge detection, image feature extraction, distance transform, skeleton extraction, cell tracking and so on. In practice, thresholding can solve most problems. However, a good thresholding value is required.

Although its simplicity, there is no strict definition for the thresholding of an image. Quite a lot of thresholding techniques \cite{sahoo1988survey, sankur2001image, sezgin2004survey}, more than 44 binary methods, are proposed according to different criterions. Sezgin and Sankur \cite{sankur2001image, sezgin2004survey} categorize thresholding methods into six groups based on different models, that is histogram shape-based methods, clustering-based methods, entropy-based methods, object attribute-based methods, spatial methods, and local methods. Histogram shape-based methods find the thresholding value on histogram data by seperating the peaks and valleys, the representing method is otsu's thresholding method. Clustering-based methods models the foreground and background as a mixture of two gaussians and applys clustering methods to get the two parts. Entropy-based methods utlize the entropy of the foreground and the background regions, as well as the cross-entropy between the original and binarized image, etc. Object attribute-based methods finds a partition which is similar to the gray-level image in some attributes, such as fuzzy shape similarity, edge coincidence, etc. The spatial methods utilize the probability distribution in higher-order and/or correlation between pixels. Local methods calculate a suitable threshold value for each pixel according to the local image characteristics, such as the standard devariance, mean, etc.

As thresholding method is not the main segmentation method in our paper, we only introduce some widely used algorithms. For global thresholding method, otsu's method \cite{otsu1975threshold} is a very elegant method with solid mathematic fomulars by minimize the intra-class variance or maximize inter-class variance. And it is easy to extend otsu's method into multi-thresholding method. For local thresholding, the thresholding is decided by average local gray values and/or standard devariance. 
\subsection{Global thresholding and Otsu's method}
Global thresholding converts a grey-level image into binary image by turning all pixels below some threshold to zero and all pixels above that threshold to one. If $g(x,y)$ is a thresholded version of $f(x,y)$ at some global threshold $T$, 
\begin{equation}
g(x,y) = \left\{
  \begin{array}{ll}
  1 & \mbox{if } f(x,y) \ge T \\
  0 & \mbox{otherwise}
  \end{array}
  \right.
\end{equation}
To set a global threshold $T$, we usually analysis the histogram profile by finding a valley that seperates two mountains. One mountain for the foreground and one for the background. The histogram of an image is a probability distribution,
\begin{equation}
p(g) = n_g/n
\end{equation}
where, $n_g$ is the number of pixels with intensity $g$, $n$ is the total number of pixels. There are two ways to decide the global threshold, the iterative method and otsu's method. 

\textbf{Iterative method :} This method compute the global threshold from the initial mean intensity value, then iterative replace the threshold value by the average mean intensity of the foreground and the background regions. See alg\ref{alg:global-thresh} for details of this method.

The main problem for the iterative method is speed. The step for segmenting an image into foreground and background for many times is time consuming.

\begin{algorithm}
\SetAlgoLined
\KwData{Grey-level image and the histogram}
\KwResult{The global threshold}
Estimate the initial threshold $T$ with the mean value.\\
Divide the image into foreground area $F$ and background area $B$.\\
Calculate the mean intensity $\mu_f$ and $\mu_b$ for area $F$ and $B$ respectively.\\
Refresh the threshold $T = (\mu_f + \mu_b)/2$\\
Repeat 2-4 until $\mu_f$ and $\mu_b$ do not change any more
\caption{Iterative method for global thresholding}
\label{alg:global-thresh}
\end{algorithm}
\textbf{Otsu's method : } Otsu \cite{otsu1975threshold} proposed a method based on selecting the lowest point between two classes. The selected point will minimize the intra-class variance or maximize the inter-class. The intra-class variance is defined as the weighted sum of variances of the foreground area and background area.
\begin{equation} \label{eq:intra-var}
\sigma_w^2(t) = w_b(t)\sigma_b^2(t) + w_f\sigma_f^2(t)
\end{equation}
where $w_f$ and $w_b$ are the probabilities of the two classes seperated by threshold $t$, $\sigma_f^2$ and $\sigma_b^2$ are the variance of foreground and background regions.

Further, Otsu demonstrate that minimizing the intra-class variance is the same as maximizing inter-class variaces:
\begin{equation} \label{eq:inter-var}
\sigma_b^2(t) = \sigma_T^2 - \sigma_w^2(t) = w_f(t)w_b(t)[\mu_f(t) - \mu_b(t)]^2
\end{equation}
where $\sigma_T^2$ is the total variance of the whole image, $\mu_f$ and $\mu_b$ are the mean intensity of foreground and background regions. Here $w_b(t) = \sum_0^tp(i)$, $w_f(t) = 1 - w_b(t)$, $\mu_b(t) = \sum_0^tp(i)x(i)/w_b(t)$, and $\mu_f = (\mu_T - \mu_b(t)w_b(t))/w_f(t)$

By using fomular \ref{eq:inter-var}, the Otsu's method could be designed as dynamic algorithm, and thus will be very fast. The equations for dynamic otsu's method is as follows, 
\begin{equation} \label{eq:otsu-dynamic}
\begin{array}{lll}
	w_b(t) & = & w_b(t-1) + p(t) \\
	w_f(t) & = & \mu_T - \mu_b(t)w_b(t) \\
	\mu_b(t) & = & (\mu_b(t-1)w_b(t-1) + p(t)x(t))/w_b(t)\\
	\mu_f(t) & = & (\mu_T - \mu_b(t)w_b(t))/w_f(t)
\end{array}
\end{equation}
Where $\mu_T$ is the average intensity of the whole image. See alg.\ref{alg:otsu-thresh} for the details. If there are multiple maximum $\sigma_b(t)^2$, the thresold value can be set as the average of them.
\begin{algorithm}
\SetAlgoLined
\KwData{Grey-level image and the histogram}
\KwResult{The global threshold}
Compute the histogram and probabilities $p(g)$ for each intensity level $g$\\
Initilize $w_i(0)$ and $\mu_i(0)$\\
Step through all possible thresholds one by one, compute $\sigma_b(t)^2$ according to eqn.\ref{eq:inter-var} and eqn.\ref{eq:otsu-dynamic}\\
Find the threshold correspond to the maximum $\sigma_b^2(t)$
\caption{Otsu's method for global thresholding}
\label{alg:otsu-thresh}
\end{algorithm}
\subsection{Local thresholding method}
The major problem with global thresholding is that it considers only the intensity, not any relationships between the pixels or any local characteric. The global thresholding can't handle changing illumination. It can give poor results for certain types of images. And the pixels identified by the thresholding process are not at all continuous. By applying local approach, we can overcome some of the problems.

Local thresholding divide an image into sub-images by a sliding window. For each sub-image, we find its global threshold. If the region is constant, consider it against a global threshold (all black or white). If there is sufficient variance, use Otsu/Iterative method in the window. 

Generally speaking, the local threshold is set according to the local mean and local variance.
\begin{equation}
T_{local} = a\cdot\mu_{local} + b\cdot\sigma_{local}
\end{equation}
in which the coefficient $a$ for $\mu_{local}$ and the coefficient $b$ for $\sigma_{local}$ is decided according the illumiation gradience or by experience.
\subsection{Component tree based thresholding} \label{sec:thresh-cptree}
When the image is slightly complex, the limitation of thresholding methods becomes very obvious. We always can't use a single threshold to get the objects we are interested, especially when there are multiply objects, where each object lies in different gray levels. In such case, some objects will get miss-segmented or half-segmented. Another example is when the background is not evenly distributed, such as vignetting background, which is due to uneven illumination, or linear background. One single threshold will inevitably divide the background area into foreground area.

Even though the local thresholding can overcome some of the problem in global thresholding, but it is not easy to decide the size of sliding windows and the coefficients. In our paper, we use a much advanced method which considers all possible thresholds. We will get all possible connected-component in different threshols. The relationship between connected-component is either inclusion or non-overlap. So we can build a tree, called component tree, to manage the relationship for all connected-component. 

With component tree, we don't have to bother about the best threshold. We can filter out the components with certain size, which is obtained from a piori knowledge. The details about component tree will be introduce in chapter \ref{chapter:cptree}.
\section{Watershed methods}
\section{Snake: Active contour method}
\subsection{Overview}
The active contour model\cite{kass1988snakes} (also called snake model) is popular in computer vision, which finds the object boundarys either continous or non-continuous. It is greatly used in application like image segmentation, object tracking, shape recognition, edge detection, and stereo matching.

A snake in the image is a lot of discrete points, which is called a spline.
\begin{equation}
v_s = (x_s, y_s)
\end{equation}
where $s \in [0,1]$. The snake is guided by many forces which maybe external forces from the image gradience or the internal forces from the snake curve itself. When balance, the snake will be guided to the image boundaris and stopped.

To describe the snake state, each state (position) of the snake has an energy, which is the sum of external energy and internal energy, corresponding to it. The internal energy $E_{int}$ of the spline (snake) is due to bending. Then external energy $E_{ext}$ consist of the image forces $E_{img}$ acting on spline and the constrainted forces $E_{con}$ introduced by user. 
\begin{equation}
E_{snake} = \int_0^1E_{snake}(v(s))ds = \int_0^1(E_{ext} + E_{int})ds \\
\end{equation}
\begin{equation}
E_{ext} = E_{img} + E_{con}
\end{equation}

\textbf{Internal energy: } The bending forces of the snake come from the curve length and curve curvature.
\begin{equation}\label{eqn:int-energy}
E_{int} = (\alpha(s)|v_s(s)|^2 + \beta(s)|v_{ss}(s)|^2)/2
\end{equation}
where $\alpha(s)$ and $\beta(s)$ controls the energy sensitive to snake stretching and curve roundness, $|v_s(s)|^2$ represents the snake length and $|v_{ss}(s)|^2$ represents the total curvature. The larger the value of $\alpha(s)$, the more sensitive of the internal energy as the snake stretches. And the larger the value of $\beta(s)$, the more sensitive of the internal enerygy as the curve increase.

\textbf{Image forces: } The forces has three components that is lines, edges and terminations.
\begin{equation}
E_{img} = w_{line}E_{line} + w_{edge}E_{edge} + w_{term}E_{term}
\end{equation}
The line component is just the intensity of the image.
\begin{equation}
E_{line} = I(x,y)
\end{equation}
Edges in the images will make the snake attract to the area with large image gradients.
\begin{equation}
E_{edge} = -|\nabla I(x,y)|^2
\end{equation}
Or the edge computed on the gaussian blured image,
\begin{equation}
E_{edge} = -|G_\sigma \cdot \nabla^2 I|^2
\end{equation}
The termination component of energy can be defined as
\begin{equation}
E_{term} = \frac{\partial \theta}{\partial n_\perp} = \frac{C_{yy}C_x^2 - 2C_{xy}C_xC_y + C_{xx}C_y^2}{(C_x^2 + C_y^2)^{3/2}}
\end{equation}
where $C(x,y)$ is the gaussian smoothed image on $I(x,y)$, $C_x$ and $C_y$ are gradient along $x$ and $y$, $C_{xx}$, $C_{xy}$, and $C_{yy}$ are second order gradients. This component is used to detect corners and terminations in an image.

\textbf{Constraint energy: } In some systems, the user interaction on the snake can guide the snake towards or away from particular features. 
\subsection{Energy minimization model}
The snake will move toward the energy decreasing direction. One of the simplest optimization method is gradient-descent minization\cite{snyman2005practical}. The energy of the snake can be estimated by using the discrete points on the snake. 
\begin{equation}
E_{snake} \approx \sum_1^nE_{snake}(\bar{v}_i)
\end{equation}
Thus the devirative of the energy is approximate to
\begin{equation}
\nabla E_{snake} \approx \sum_1^n \nabla E_{snake}(\bar{v}_i)
\end{equation}
Now applying the gradient descent minimization, the position of the snake is adjust as,
\begin{equation}
\bar{v}_i = \bar{v}_i - \nabla E_{snake}(\bar{v}_i)
\end{equation}
This equation is applied iteratively until the energy doesn't change.
\subsection{Other implementations}
The classic snake modes which driven the snake towards object contours depends very much on the initial snake position.
The initial snake position should be close to the object or intercross with object boundary. Otherwise, the snake may
get stuck in local minima states and mis-finding the object. To overcome the initial position problem, many variations
based on snake model are proposed.

\textbf{GVF active contours: } In normal snake modes, the diffuse forces exist only on the object boundary. This 
makes the snake hard to converge to the object boundary. Due to this reason, the GVF active contour proposed a way to diffuse
the edge force to its surrounding, which provides a new external field. Under such model, even if the snake is started far from the object, it still gets attracted 
towards the object\cite{xu1998snakes}.

\textbf{Balloon snake: } The ballon snake \cite{kass1988snakes} behaves like a blowing ballnoon. When it passes by strong contour, it stops. Thus, the
initial snake need not to be too close to the object. The external field is modified and a new pressure force is introduced to make the curve evolve like a ballon.

\textbf{Diffusion snakes: } The diffusion snake \cite{cremers2002diffusion} is modified from Mumford-Shah \cite{mumford2006optimal} for spline contours. It is used 
when the prior shape information is known. The obtained segmentation maximizes both the Grey value homogeneity in the separated regions and the similarity of the contour with respect to a set of training shapes.

\textbf{Geometric Active Contours: } Geometric active \cite{malladi1995shape} proposed an energy function by summing up the snake perimeter and an inversed edge function. Through minimizing the energy function, the snake moves towards the perimeter shrinking direction and stops at the high edge area. The model may be implemented using level sets, which will be introduced in next section. For the model, the snake can be initialized relatively large to enclose the objects.
\section{Level set based segmentation}
Level set method is a numerical method for the evolution of curve (or interface). In mathematics, a level set of a function $f$ with $n$ variables is the set of the form,
\begin{equation}
LS_c(f) = \{(x_1,\ldots,x_n)|f(x_1,\ldots,x_n) = c\}
\end{equation}
where $c$ is the level set value. The level set can be used to represent a curve implicitly. For example, the two dimension circle $x^2 + y^2 = 1$ can be considered as the zero level of function $f(x,y) = x^2 + y^2 - 1$. With level set function, we can easiy define the inside area, background area and interface $\Gamma$.
\begin{equation}
\Gamma = \{\vec{x}|f(\vec{x}) = 0\}
\end{equation}
\begin{equation}
Inside = \{\vec{x}|f(\vec{x}) > 0\}
\end{equation}
\begin{equation}
Outside = \{\vec{x}|f(\vec{x}) < 0\}
\end{equation}

The great advantage of level set methods is the ability to utilize the regional information, such as the area and average intensity properties. Based on such merits, Chan and Vese \cite{chan2001active,chan2000active} proposed a modified Mumford-Shah function to segment an image into piecewise constant regions without considering the edge information. However, Chan-Vese model can only segment an image into two phase (two distinct regions) which is hard to distinguish multi-objects. To break through this limitation, a multi-phase level set method \cite{vese2002multiphase} is proposed to segment an image into $N$ phases with $log_2N$ level set functions, where each phase represents an object. And further, to overcome the cell touching problem, the coupled level set model is proposed by Zhang \cite{zhang2004tracking} to segment $N$ objects with $N$ level set functions. Dufour \cite{dufour2005segmenting} extends the model to 3D confocal image segmentation. Palaniappan \cite{nath2006robust} did the work of lower down the number of level set functions to only four.

\subsection{Explicit methods vs. implicit methods}
Usually there are two ways to represent a curve, explicitly or implicitly. The explicit way defines a curve with parameter, 
\begin{equation}
C = \{C_i| C_i=(x_i, y_i), i \in \{1,\ldots,n\}\}
\end{equation}
where $n$ is the vertex number on the curve, $C_0$ is the start point and $C_n$ the last point. The implicit way defines a curve as the zero level set of a function,
\begin{equation}
C = \{(x,y)|f(x,y) = 0\}
\end{equation}
The solution for an explicit curve is explicit method, and the solution for an implicit curve is implicit method. 

\emph{Explicit methods} calculate the state of a curve at a later time from the state of the curve at the current time, while \emph{implicit methods} find a solution by solving an equation involving both the current state of the curve and the later one. Let $C(t)$ stands for current curve state and $C(t+\Delta t)$ is the state at the later time, then , for an explict method
\begin{equation}
C(t+\Delta t) = F(C(t))
\end{equation}
while for an implicit method one solves an equation
\begin{equation}
G(C(t), C(t+\Delta t)) = 0
\end{equation}
to find $C(t + \Delta t)$.

Obviously, the implicit method will need more computation time and hard to implement. However, with level set method the implicit methods have many great feature. Without having to parameterize these objects, the level set method enables an easy way to follow shapes that change topology, such as shape spliting, shape merging, and adding holes. As many fast algorithms appeared, the limitation of level set method becomes less important. 

For some segmentation problems, such as Chan-Vese model bellow, it impossible to use explicit methods when the regional information is considered, while implicit methods solve the problem naturally.
\subsection{Chan-Vese model}
Based on the level set curve evolution and Mumford-Shah function, Chan and Vese \cite{chan2001active} proposed a new model for active contours to detect objects in an image. The model can detect objects whose boundary is not so called sharp edges. Though without edges, the evolving curve can still stop at the desired position by minimize an energy function. With level set technique, Chan-Vese model overcomes many difficulties arising in previous methods (snake model) of image segmentation.

Chan-Vese model considers the segmentation result of an image $u_0$  as a piecewise constant image $u$. The foreground (inside) area of $u$ is of constant intensity $c_i$, and the background (outside) area of $u$ is of constant intensity $c_o$. The energy function for the segmentation result $u$ is defined as the difference between $u_0$ and $u$.
\begin{eqnarray*}
E(C) & = &  \int_{inside( C)}|u_0(x,y) - c_i|^2dxdy + \int_{outside( C)}|u_0(x,y) - c_o|^2dxdy
\end{eqnarray*}
Here $C$ is the segmentation contour (curve), which is the zero level set of a function $\phi$,
\begin{equation}
C = \{(x,y) \in \Omega|\phi(x,y) = 0\}
\end{equation}
And the inside and outside area is defined as the positive level sets and negative level sets of $\phi$.
\begin{equation}
inside( C) = \{(x,y) \in \Omega|\phi(x,y) > 0\}
\end{equation}
\begin{equation}
outside( C) = \{(x,y) \in \Omega|\phi(x,y) < 0\}
\end{equation}
For the constant values $c_i$ and $c_o$,  they can be user defined intensity or the average intensity of inside and outside area,
\begin{equation}
c_i(\phi) = \frac{\int_{\Omega}u_0(x,y)H(\phi(x,y))dxdy}{\int_{\Omega}H(\phi(x,y))dxdy}
\end{equation}
\begin{equation}
c_o(\phi) = \frac{\int_{\Omega}u_0(x,y)(1-H(\phi(x,y)))dxdy}{\int_{\Omega}(1-H(\phi(x,y)))dxdy}
\end{equation}
where $H(z)$ is a binary function, that is one for positive value and zero for negative value.

To get a more smooth boundary, we can add more regularizing items to the energy function,
\begin{eqnarray}
\label{eqn:chanvese}
E(c_i,c_o,C) & = & \mu\cdot\mbox{Length( C)} + \nu \cdot \mbox{Area} + \lambda_1\int_{inside( C)}|u_0(x,y) - c_i|^2dxdy \\
\nonumber
& & + \lambda_2\int_{outside( C)}|u_0(x,y) - c_o|^2dxdy
\end{eqnarray}

This energy function is an extension of geometric active contour model by minimize the perimeter of the contour. Here the length and area energies are defined as, 
\begin{equation}
\mbox{Length}\{\phi = 0\} = \int_\Omega|\nabla H(\phi(x,y))|dxdy = \int_{\Omega}\delta_0(\phi(x,y))|\nabla\phi(x,y)|dxdy
\end{equation}
\begin{equation}
\mbox{Area}\{\phi \ge 0\} = \int_{\Omega}H(\phi(x,y))dxdy
\end{equation}
where $\delta_0(z) = \frac{d}{dz}H(z)$ and $\Omega$ is the image domain.

\subsubsection{Numerical solution}
Equation\ref{eqn:chanvese} is solved by variational method. Through substituing into the Euler-Lagrange equation and applying Green's identity and Green's theorem, the solution PDE is,
\begin{equation}
\frac{\partial \phi}{\partial t} = \delta(\phi)\left[\mu\mbox{div}\left(\frac{\nabla\phi}{|\nabla\phi|}\right)-\nu -\lambda_1(u_0-c_1)^2 + \lambda_2(u_0-c_2)^2\right]
\end{equation}

\subsubsection{Effect of weights}
Each energy item in eqn.\ref{eqn:chanvese} is weighted according to real instance. For length energy weight, $\mu$ controls the smoothness of the contour. High $\mu$ value enables a tightly attached contour to the object, whereas low $\mu$ value enables a loosely enclosed contour. The area weight $\nu$ inhibit the growing of inside area. The relative balance between $\lambda_1$ and $\lambda_2$ determines which side, inside or outside, has higher importance in minimizing the energy. 

\subsubsection{Generalizing to N-dimension}
Chan-Vese model can be naturally extended to N-dimensional image segmentation with the energy function,
\begin{eqnarray}
\nonumber
E(\phi, c_i, c_o) & = & \mu\int_\Omega |\nabla H(\phi(\vec{x}))|d\vec{x} + \nu\int_\Omega H(\phi(\vec{x}))d\vec{x}\\
                 &    & + \lambda_1\int_\Omega |u_0(\vec{x} - c_i)|^2H(\phi(\vec{x}))d\vec{x} \\
\nonumber
				 &    & + \lambda_2\int_\Omega |u_0(\vec{x} - c_o)|^2(1 - H(\phi(\vec{x})))d\vec{x}
\end{eqnarray}
And the solution is similarly,
\begin{equation}
\frac{\partial \phi(\vec{x})}{\partial t} = \delta_0(\phi(\vec{x})\left[\mu\mbox{div}\left(\frac{\nabla\phi(\vec{x})}{|\nabla\phi(\vec{x})|}\right)-\nu -\lambda_1(u_0(\vec{x})-c_i)^2 + \lambda_2(u_0(\vec{x})-c_o)^2\right] 
\end{equation}

\subsubsection{Generalizing to vector-valued images}
For an image with multi-channels (RGB image), each pixel is a vectorized value. The energy is defined as the average energy of each channel \cite{chan2000active},
\begin{eqnarray}
\nonumber
E(\phi,\vec{c^+}, \vec{c^-},\phi) & = &\mu\cdot L + \int_{inside( C)} \frac{1}{N}\sum_{i=1}^N\lambda_i^+|I_i(x,y)-c_i^+|^2dxdy \\
& & + \int_{outside(C )}\frac{1}{N}\sum_{i=1}^N\lambda_i^-|I_i(x,y)-c_i^-|^2dxdy
\end{eqnarray}
where $\vec{c^+}$ and $\vec{c^-}$ are inside average intensity and outside average intensity for each channel with the same segmentation result $\phi = 0$. And the corresponding solution,
\begin{equation}
\frac{\partial \phi}{\partial t} = \delta(\phi)\left[\mu\mbox{div}\left(\frac{\nabla\phi}{|\nabla\phi|}\right) -\frac{1}{N}\sum_{i=1}^N\lambda_i^+(I_i(x,y)-c_i^+)^2 + \frac{1}{N}\sum_{i=1}^N\lambda_i^-(I_i(x,y)-c_i^-)^2\right]
\end{equation}

For vector-valued ChanVese model, besides its application to RGB images, it can be used for texture image segmentation. A texture image can be decomposited into many small blocks and each block is calculated with many features. Thereby a texture image can be converted into a low resolution image with many channels (that is one feature one channel).

\subsection{Multi-objects segmentation}
Chan-Vese model segment an image into two phases, the interested cells are all segmented into one region, which is hard to distinguished. Later a multiphase variant of Chan-Vese model was also proposed to handle $2^n$ unique phases with $n$ level set functions. But Zhang \cite{zhang2004tracking}, Dufour \cite{dufour2005segmenting} and Zimmer \cite{zimmer2005coupled} found that the multi-phase model is unsuitable for realiable cell segmentation due to the problem of frequently cell merging. By considering the merging event, Zhang \cite{zhang2004tracking} proposed an $N$-level set framework, that each level set function describes an object, by adding a coupling punishment for merging areas. The following two sections will introduce multi-phase level set methods and the coupling level sets method.

However, during my experiment I find it not too much help with coupling constraints. It's hard to set the coupling weight. The coupling area usually changes dramaticly. For an un-suitable coupling weight, the coupling area dispears rapidly once two cells merged together. For a reason, we can force the level set function stop growing when meet with the boundary of another level set function. So we can set a boundary for each level set function according to the initial object area. Such bundary could actually be obtained from a generalized voronoi diagram of all input objects. After that, we can perform Chan-Vese's level set method in each voroni area. The voronoi diagram changes every a few iterations according the last level set segmentation results. With voronoi diagram, we will need only one level set function to segment any number of objects. 

\subsection{Multiphase level sets method}
To segment multiple objects, Chan and Vese extended their 2-phase algorithm to N-phase algorithm by using $log_2N$ level set function. Because, $n$ level set functions could form $2^n$ intercrossing region. Each 
\subsection{Coupled N level sets method}
The number of coupling level set functions can be reduced from N to 4 \cite{nath2006robust} by using coloring theory. level set functions.

\subsection{V-D based one level set method}
The generalized voronoi diagram is introduced in Sec.\ref{subsec:gvd}
\section{Graph cut method}
\section{Open-source librarys}
\subsection{ITK}
\subsection{OpenCV}
\subsection{My work}
 % chapter1: image segmentation
\chapter{Component Tree}\label{chapter:cptree}
 % chapter2: component tree
\chapter{Tree assignment and tracking results} \label{chpt:treeassign}
Our cell tracking method needs to generate a bipartite matching \cite{mosig2009tracking, Xiao:2011} between consective component trees, which performs the role of both cell segmentation and cell association. In the most general form, generalizations of bipartite matching to trees have recently been shown to be $\mathcal{NP}$-hard\cite{canzar2011tree}.
In this thesis, we introduced two types of tree assignment problems. One is tree assignment between component trees, the other is tree assignment between neuron trees. Both models can be implemented by integer linear programming(ILP) and dynamic programming(DP). The two types of solution have different running time which depends on the complexity of tree structure. The tree assignment between trees with each node of few child nodes, such as component tree, is suitable to use DP, otherwise it is suitalbe to use ILP.

\section{Component tree assignment}
\subsection{Problem formulation}
Let $T_1$ and $T_2$ denote two rooted unordered trees, with vertices $U$ and $V$, respectively.  We refer to the set of all possible assignments between $T_1$ and $T_2$ as match $(T_1,T_2) =\Big\{M \subset U \times V | M = \{(u_1,v_1),\ldots,(u_k,v_k)\}\Big\}$. Given a weighting function $w:U\times V \to \mathbb{R}_{\le 0}$, we can assign a weight $W(M) := \sum_{u,v)\in M}w_{u,v}$ to an assisnment $M \in$ match$(T_1,T_2)$. Putting things together, this allows us to define the \emph{tree assignment problem}, which is to find the maximum weighted tree assignment $M$, given $T_1,T_2$ and $w$. The tree assignment problem is a generalization of maximum weighted bipartite matching problem.
\subsubsection{Three points condition}
We deal with tree assignments under \emph{three points condition} for component tree assignment base cosegmentation, which introduces a restriction on the topology of trees with three leaves. We adapt the recursions for the constrained tree edit distance to solve the restricted tree assignment problem.

The three-point condition involves the lowest common ancestor of two vertices $a,b$ in a tree, which we denote by lca$(a,b)$. Now, we defined the tree assignment matching $M \in$ match$(T_1,T_2)$ if for any tree assignments $(a_1,a_2),(b_1,b_2),(c_1,c_2) \in M$, we have
$$
lca(a_1,b_1) \preceq lca(a_1,c_1)  \Rightarrow lca(a_2,b_2) \preceq lca(a_2,c_2)
$$

Intuitively, the three-point condition ensures that for any tree pairs of vertices in an assignment, the topology of the two induced subtrees are in the same heirarchical order.  
\subsubsection{Component tree}
The component tree maintains the inclusion relationship between all possible connected component under all different threshold segmentation. Najman et al.\cite{Najman:04} introduced a quasi-linear algorithm to construct the component tree. And I introduced a pruning principle to simpify the component tree structure (see Chapter\ref{chpt:cptree}).

\subsubsection{Component tree weights}
The weight between two component nodes is defined as the ratio of overlapping size and union size. I introduced a quasi-linear dynamic algorithm to compute all the weights between two component trees (see Sec\ref{sec:cptree-weight}).
\subsection{Integer linear programming (ILP)}
Given two trees $T_1$ and $T_2$, to satisfy the \emph{three point condition}, we introduce the match $M \in$ match$(T_1,T_2)$ such that for any two distinct indices $1 \le i,j \le k$, neither $u_i$ is an ancestor/descendant of $u_j$ nor $v_i$ is an ancestor/descendant of $v_j$. 

For each vertex $u \in T_1, v \in T_2$, introduce binary variable $X_{uv}$, where $X_{uv} = 1$ iff $u$ is assigned to $v$, otherwise $X_{uv} = 0$. For any node $u, u' \in T_1$ and $v, v' \in T_2$, we can define the integer linear model with constraints,
\begin{equation}
\label{eqn:cptree-ilp-st}
\left\{
\begin{array}{l}
X(u,v) \in \{0,1\} \\
\forall u \preceq u', X(u,v) + X(u',v') \le 1 \\
\forall v \preceq v', X(u,v) + X(u',v') \le 1 
\end{array}
\right.
\end{equation}
and the objective function by maximum the sum of weight function,
\begin{equation}
 \max \sum_{u\in T_1, v \in T_2} w(u,v)X(u,v)
\end{equation}

For the constraints \ref{eqn:cptree-ilp-st}, the first constraint ensures the binary values for assignment variable, the second and third constraint ensure that only one node is matched in a leaf-node-to-root-node path in both trees (see Fig\ref{fig:treeassign-cptree}A and Fig\ref{fig:treeassign-cptree}B). 

Let $L(T)$ denotes all the leaf nodes of T; $root(T)$ denotes the root node of tree $T$; $path(u,v) = \{u_1 = u,u_2, \ldots, u_{n-1}, u_n = v\}$ denotes the path from node $u$ to node $v$; $P(T) = \{path(u, root(T))|u \in L(T)\}$ denotes all the leaf-node-to-root-node paths. The above constraints \ref{eqn:cptree-ilp-st} can be re-written as,
\begin{equation}
\left\{
\begin{array}{l}
X(u,v) \in \{0,1\} \\
\forall p \in P(T_1), \sum_{u\in p}\sum_{v \in T_2} X(u,v) \le 1 \\
\forall p \in P(T_2), \sum_{u\in T_1}\sum_{v \in p} X(u,v) \le 1 
\end{array}
\right.
\end{equation}

\subsection{Dynamic programming algorithms (DP)}


Jiang \cite{Jiang:95} provided a less constraint tree assignment algorithm.
For each $A \subseteq \{u_1,u_2,...,u_m\}$, $u_i \in T_1$ and each $B \subseteq \{v_1,v_2,...,v_n\}, v_j \in T_2$. \\
\begin{eqnarray} \label{eqn:lcdp-formular}
W(A,B) = \max \begin{cases}
\max_{a_i \in A, b_j \in B}(W(A-\{a_i\}, B-\{b_j\}) + w(a_i,b_j))\\
\max_{a_i \in A, B' \subseteq B}(W(A-\{a_i\}, B-B') + W(C(a_i), B')) \\
\max_{A' \subseteq A, b_j \in B}(W(A-A', B-\{b_j\}) + W(A', C(b_j))
\end{cases} 
\end{eqnarray}

Assume $r_1$ the root node of $T_1$ and $r_2$ the root of $T_2$, the solution of tree assignment will be $W(C(r_1), C(r_2))$.
Obviously the constrained tree assignment in section \ref{sect:tree-assignment} is covered as a special case by applying the first case of Eqn.\eqref{eqn:lcdp-formular}. 

\begin{figure}[htbp]
\centering
\includegraphics[width=1.0\textwidth]{images/treeassign_cptree}
\caption[ILP and DP models for component tree assignment]{ILP and DP models for component tree assignment. A. The second constraint in ILP model \ref{eqn:cptree-ilp-st} for component tree assignment; B. The third constraint in ILP model \ref{eqn:cptree-ilp-st} for component tree assignment; C. The DP model of component tree assignment.}
\label{fig:treeassign-cptree}
\end{figure}

\subsubsection{Running time}
With the less constrained recurrence, we will get much better result with higher mapping score. However, the time complexity turns out to depend exponentially on the maximum degree of the two trees
\begin{equation*} \label{eqn:lcdp-time}
O\left(\sum_{u_i \in T_1, v_j \in T_2}(2^{|C(u_i)|}\cdot 2^{|C(v_j)|})\right).
\end{equation*}
For binary trees, the running time will be
$O(|T_1|\cdot|T_2)$. \\
For component trees, the runing time will be 
$O\left(2^{|C(r_1)|}\cdot2^{|C(r_2)|}\right)$. \\
\subsubsection{Improvement}
Firstly we define the overlap between $A$ and $B$. We mean $A$ and $B$ have overlap that there exists some element in $A$ ovelaping with some element in $B$, otherwise no overlap.\\ 
When dealing with tree-assignments involving component trees, note that in practice many $A$ and $B$ have no overlap, or only few elements in $A$ overlap with few elements in $B$. The non-overlaping A and B can be neglected in the assignment. The overlaping A and B can be decomposed into disjoint subsets $A=A_0 \cup \ldots \cup A_n$ and $B = B_0 \cup \ldots \cup B_n$ such that $A_1 \cup B_1, \ldots, A_n \cup B_n$ are the connected components in graph $G(V, E)$ , that $V = A \cup B$ and $E = \{(a,b)|a \in A, b \in B, \beta(a) \cap \beta(b) \neq \emptyset\}$,  and $A_0$ and $B_0$ are the collection of isolated nodes of $A$ and $B$. As $A_0$ and $B_0$ can be neglected in the assignment, we have 
\begin{equation*} \label{eqn:ldcp-decomp}
W(A,B) = \sum_{k=1,\ldots,n}{W(A_k, B_k)}
\end{equation*}
In practice, $|A_k|$ and $|B_k|$ will be much smaller than $|A|$ and $|B|$, the runing time will be much smaller, which speed up the algorithm significantly on many instances.
\subsection{Difference between ILP and DP models}
Integer linear programming (ILP) and dynamic programming (DP) provide two different implementations for component tree assignment. DP model implements exactly the three point condition, three point condition fits the ILP model. However, ILP model doesn't always fit three point condition. Fig\ref{fig:treeassign-threepoint}A illustrates the match result in ILP which doesn't fit the three point condition. Fig\ref{fig:treeassign-threepoint2} shows the match difference between ILP and DP, where the segmentation results if ILP contain that of DP.

\begin{figure}[htbp]
\centering
\includegraphics[width=1.0\textwidth]{images/treeassign_threepoint}
\caption[Three point condition]{A. The difference between ILP and DP models, where DP fits exactly three point model, while ILP extends three point condition; B. The explaination of three point condition which is especially efficient when background gradient.}
\label{fig:treeassign-threepoint}
\end{figure}

\begin{figure}[htbp]
\centering
\includegraphics[width=1.0\textwidth]{images/treeassign_threepoint2}
\caption[ILP and DP result difference for microglia cell segmentation]{ILP and DP result difference for microglia cell segmentation, the red circles indicate the additional match result of ILP model.}
\label{fig:treeassign-threepoint2}
\end{figure}

\subsection{A model for cell spliting and merging}
The proposed algorithms can only track the movement of cell without cell splitting and cell merging, as the algorithms descirbed above assign the node in one tree to exactly only one node in the other tree. Both ILP model and DP model will loose the track of cells if cell splitting or cell merging exist. (Usually only cell splitting exists in practice.) We improve the linear tree assignment model that each node in one tree can be assigned to many nodes in the other tree. However, the constraint that only one node is assigned in a leaf-to-root-node path is kept. 

Given two trees $T_1$ and $T_2$, for each vertex $u \in T_1$, $v \in T_2$, we introduce additional two binary variables $Y_u$ and $Z_v$ besides $X_{uv}$. $X_{uv} = 1$ if node $u$ is assigned to node $v$, otherwise $X_{uv} = 0$. If node $u$ is assigned to any node in $T_2$, $Y_u = 1$; and $Y_u = 0$ if node $u$ is not assigned. The same rule works for $Z_v$. The relationship between $Y_u$,$Z_v$ and $X_{uv}$ is given by,
\begin{equation}
	Y_u = 
	\left\{
	\begin{array}{cc}
		0 & \text{if \quad $\sum\limits_{v \in T_2}X_{uv} = 0$} \\
		1 & \text{if \quad $\sum\limits_{v \in T_2}X_{uv} \ge 1$}
	\end{array}
	\right.
	\quad \text{for $u \in T_1$}
\end{equation}
\begin{equation}
	Z_v = 
	\left\{
	\begin{array}{cc}
		0 & \text{if \quad $\sum\limits_{u \in T_1}X_{uv} = 0$} \\
		1 & \text{if \quad $\sum\limits_{u \in T_1}X_{uv} \ge 1$}
	\end{array}
	\right.
	\quad \text{for $v \in T_2$}
\end{equation}
By considering the condition that only one node is assigned for a leaf-node-to-root-node path, the integer linear model is given by
\begin{equation}
\left\{
\begin{array}{l}
\sum\limits_{u \in P1 \in P(T_1)}Y_u \le  1 \\
\sum\limits_{v \in P2 \in P(T_2)}Z_v \le  1 
\end{array}
\right.
\end{equation}
with the same objective function
\begin{equation}
max\sum\limits_{i \in T_1}\sum\limits_{j \in T_2}w_{ij}X_{ij}
\end{equation}

The constraints for $Y_i$ and $Z_j$ is not clear, which seems like a non-linear interger programming. However there are two kinds of methods to define $Y_i$ and $Z_j$ and turn the problem into linear integer programming.

\textbf{Method 1:}
\begin{eqnarray}
\frac{\sum_{v \in T_2}X_{uv}}{|T_2|} \le Y_u \le \sum_{v \in T_2}X_{uv} & \text{for $u \in T_1$} \\
\frac{\sum_{u \in T_1}X_{uv}}{|T_1|} \le Z_v \le \sum_{u \in T_1}X_{uv} & \text{for $v \in T_2$}
\end{eqnarray}
Obviously when $\sum\limits_{u \in T_1}X_{uv}$ equals to zero, $Y_u$ will less than zero, due to the \{0,1\} constraint, $Y_u$ will be zero. When $\sum\limits_{u \in T_1}X_{uv}$ larger than zero,beacuse of $\sum\limits_{u \in T_1}X_{uv}$, $Y_v$ always no greater than $|T_1|$, $Y_u$ will larger then a value, which is larger then zero and no greater than 1. So $Y_u$ will of course be one. This contrains fit the definition of $Y_u$ very well. The contraint also fit to $Z_v$ very well.

\textbf{Method 2:}
\begin{eqnarray}
Y_u \ge X_{uv} &\text{for each $v \in T_2$, for $u \in T_1$}\\
Z_v \ge X_{uv} & \text{for each $u \in T_1$, for $v \in T_2$}\\
\label{con:multi-align1}
Y_u \le \sum\limits_{u \in T_1}X_{uv}  & \text{for $u \in T_1$}\\
\label{con:multi-align2}
Z_v \le \sum\limits_{v \in T_2}X_{uv}  & \text{for $v \in T_2$}
\end{eqnarray}
Once node $u \in T_1$ and node $v \in T_2$ is matched, $X_{uv}$ will be one, and $Y_u$ and $Z_v$ will be no less than $X_{uv}$, so $Y_u$ and $Z_v$ both will be 1. The constraints \ref{con:multi-align1} and \ref{con:multi-align2} is used for nonmatching instances. If node $u \in T_1$ or node $v \in T_2$ is not matched at all, $Y_u$ or $Z_v$ will be 0 either. So now the method 2 fit the definition of $Y_u$ and $Z_v$ perfectly.

\section{Neuron tree assignment}
\subsection{Two points condition}
\subsection{Optimization formula}
\subsection{Integer linear programming}
\subsection{Dynamic programming algorithms}
\section{Cell Tracking results}
 % chapter3: Tree assignment methods
\chapter{Co-segmentation based cell tracking} \label{chpt:coseg}
Cell tracking is an important method to quantitatively analyze time-lapse microscopy data. While numerous methods and tools exist for tracking cells in 2D time-lapse images, only few and very application-specific tracking tools are available for 3D time- lapse images, which is of high relevance in immunoimaging, in particular for studying the motility of microglia in vivo.

We introduce a novel algorithm for tracking cells in 3D time-lapse microscopy data, based on computing cosegmentations between component trees representing individual time frames using the so-called tree-assignments. For the first time, our method allows to track microglia in three dimensional confocal time-lapse microscopy images. We also evaluate our method on synthetically generated data, demonstrating that our algorithm is robust even in the presence of different types of inhomogeneous background noise.

Parts of this chapter follow closely the presentation from \cite{Xiao:2011}, where results of this thesis have been published.

\section{Introduction}

Capturing the motility of cells using time-lapse microscopy has become
an important approach to understanding processes such as the cell
cycle \cite{Harder:09}, neuronal division and migration
\cite{Norden:09}, immune response \cite{Cahalan:08}, or the
development of cancer \cite{Ianzini:09}. Based on phase-contrast,
confocal, or two-photon microscopy, such \emph{live cell imaging}
protocols are now commonly established and corresponding equipment is
commonly available. This has triggered the need for computational
methods to quantitatively analyze time-lapse microscopy data. In this
context, identifying individual cells and tracking their identities
over time is one of the basic ingredients for computational
analysis. Hence, \emph{cell tracking} algorithms have attracted
considerable attention in recent years
\cite{Meijering:06,Miura:05}. Here, we introduce a novel algorithm
for cell tracking that allows to track cells, in particular zebrafish
microglia, in three dimensional two-photon image sequences over time.

The majority of cell tracking algorithms, as surveyed by
\cite{Meijering:06} or \cite{Miura:05}, deals with cell tracking in
2D over time. Methods range from linking cells identified in
individual frames using different segmentation approaches to
active-contour \cite{dufour2005segmenting,Shen:06,Sacan:08} or level-set
algorithms \cite{Mukherjee:04,Nath:06,Dzyubachyk:08,Li:08}. The
challenges imposed by the nature of the images to be analyzed lie in
phenomena such as cell divisions \cite{AlKofahi:06,Li:08b}, cells
entering or leaving the displayed area, or a large number of cells
that needs to be tracked simultaneously. In addition, cell tracking is
often complicated by background inhomogeneity, for instance due to
uneven illumination \cite{Leong:03}, and cells touching each
other. While these issues have been addressed extensively for tracking
cells in 2D, surprisingly few approaches have addressed cell tracking
in 3D. Besides naive thresholding approaches, there are only few
advanced approaches, such as the active-contour based method proposed
by \cite{dufour2005segmenting}. Recently, several authors
\cite{Jaensch:10,Kerekes:09} proposed reliable methods for tracking
centrosomes in \emph{C.elegans} embryos. Yet, these approaches are
tailored towards tracking small, bright, and circular objects which
e.g. resemble a Gaussian spot of a specific size. Such assumptions,
however, are not satisfied by the complex and highly variable shapes
of microglia under consideration here. Cell tracking is also relevant
in the context of tracking cell populations \textit{in vitro}, which
has attracted considerable attention recently
\cite{House:09,Padfield:09,Ong:10}.

\begin{figure}
  \centering
  \includegraphics[width=.7\linewidth]{images/coseg_fig1}
  \caption[3D motion patterns of two microglia \textit{in vivo} 
   reconstructed using \texttt{ct3d}]{3D motion patterns of two microglia \textit{in vivo}
    reconstructed using \texttt{ct3d}.  The red and green areas
    indicate initial positions of the two microglia. While the red
    cell remains in resting state, the green cell is activated through
    an induced injury and migrates along a trajectory (orange line:
    trajectory obtained by \texttt{ct3d}; blue line: trajectory
    obtained from manual annotation) towards a site of injury (purple
    dot);}
  \label{fig:coseg-fig1}
\end{figure}

The lack of methods for tracking cells in 3D has been reported as a
limiting factor, for instance in the context of immunoimaging
\cite{Cahalan:08}. Despite the well-established protocol to capture
microglia, innate immune cells in the central nervous system, in 3D
using two-photon microscopy following the seminal works by
\cite{Nimmerjahn:05} and \cite{Davalos:05}, motility analysis has
been performed by (and limited to) manual estimations derived from 2D
projections \cite{Davalos:08} in the numerous studies following these
protocols. In fact, tracking microglia cells is complicated by several
aspects. Microglia tightly contact specific brain structures in their
resting state \cite{Wake:09}, often making it difficult to clearly
separate them from their surrounding tissue. Furthermore, the
extension and retraction of so-called microglia \emph{processes} makes
it practically impossible to separate them from other cells or
surrounding tissue in a 2D projection. As we demonstrate in this
study, cosegmentation based cell tracking may overcome these
difficulties and allows to reliably track microglia in 3D, both in
resting state and when moving in activated state, as displayed in
Figure \ref{fig:coseg-fig1}.

% Our method an related approaches: thresholding / localized /
% mutli-thresholding
From an algorithmic point of view, our method can be seen as a broad
generalization of \emph{thresholding methods}. Otsu's early and still
commonly used approach \cite{otsu1975threshold} picks a cut-off intensity based
on the gray-value histogram of an image, considering pixel intensities
below this threshold as background and pixels exceeding the threshold
intensity as foreground. To deal with background inhomogeneities and
objects of varying intensities, different approaches such as locally
adaptive thresholding \cite{Kim:05} have been developed. Our approach
utilizes a highly systematic way of picking local thresholds in a
hierarchical representation of all possible thresholds of an image,
the so-called component tree \cite{jones1999connected,Najman:04}. In order to
pick local thresholds in the component tree, we compare the component
trees of consecutive time frames by solving the so-called \emph{tree
  assignment} problem, a natural generalization of bipartite matchings
and the associated assignment problem. Comparing component trees by
computing tree assignments yields a \emph{cosegmentation} of two
images; for cell tracking, cosegmentations between two time frames in
a video sequence are of particular relevance.

While the term cosegmentation has been coined by \cite{Rother:06},
our approach significantly differs from their approach, which is based
on comparing histograms. On the contrary, our approach is
morphological in the sense that it attempts to identify overlapping
regions in two images by finding an optimal tree assignment.

Using cosegmentation has potential further applications in location
proteomics beyond the cell tracking problem investigated in this
paper. Tree assignments as a generalization of bipartite matchings
were introduced and applied by the last author recently
\cite{mosig2009tracking}, and were recently shown to be computationally hard
in general \cite{Klau:10}. Applying tree assignments to component
trees for obtaining cosegmentations is a novel contribution in this
work. In fact, cosegmentations promise to be useful in other
bioimaging (and eventually image processing) applications beyond cell
tracking. One straightforward application where cosegmentation is of
high relevance are protein-colocalization studies. Studying
colocalization has recently become of relevance through the
availability of corresponding two- or multi-label fluorescence
microscopy \cite{Zinchuk:08,Schubert:06} or in-situ hybridization
\cite{Carson:09,Boettiger:09} techniques.

We implemented our algorithm in the publicly available \texttt{ct3d}
software package, which is accompanied by the \texttt{at3d} graphical
user interface. In terms of applying our algorithm, this paper focuses
on evaluating the performance of our cosegmentation based approach for
3D cell tracking, leaving colocalization studies as a future
direction. Cell tracking performance is evaluated both on two-photon
live cell imaging data displaying zebrafish microglia \textit{in
  vivo}, and on synthetically generated data that allow to determine
the algorithm's accuracy based on the ground truth the synthetic data
were generated from.

\section{Methods}

\begin{figure}[htbp]
\centering
\includegraphics[width=.5\textwidth]{images/coseg_fig2}
\caption[Tree assignment of two (pruned) component trees for two 1D images I and J]{Tree assignment of two (pruned) component trees for two 1D images I and J. Vertices not eliminated by the second pruning step are indicated by circles. All other non-branching vertices are eliminated in the pruning step. The tree assignment indicated by the dashed arrows is $A = \{(a, c), (b, d), (e, f)\}$ with a weight of $w_{ac}+w_{bd}+w_{ef}$.}
\label{fig:coseg-fig2}
\end{figure}

Our algorithm is based on representing each image $F_1,\ldots,F_N$ by its component tree\cite{jones1999connected}. The component tree of an image $I$ is obtained by considering the connected components of the thresholded versions $I_θ$ of I under all possible thresholds $θ$. The set of all connected components under all thresholds is obviously hierarchically ordered by subset inclusion. This hierarchical order defines the component tree, which can be computed in linear time \cite{Najman:04}. For examples of 1D images and their component trees refer to Figure \ref{fig:coseg-fig2}. 

Figure \ref{fig:coseg-fig3} illustrates the basic steps of our cell tracking algorithm. The outline of the algorithm is as follows: we start with computing and pruning component trees for each time frame. Then, tree assignments between each pair of consecutive component trees are computed. The tree assignments can be turned into segmentations of the original images. This produces two segmentations of each image, requiring computation of a consensus segmentation. The resulting unique segmentation of each image then requires a standard bipartite matching between consecutive time frames to track cell identities over time. 

\subsection{Building component trees and tree assignment}
Details for \emph{component tree construction} and \emph{tree assignment} is introduced in Chapter \ref{chpt:cptree} and Chapter \ref{chpt:treeassign}.

\begin{figure}[htbp]
\centering
\includegraphics[width=1.0\textwidth]{images/coseg_fig3}
\caption{Overview of complete cell tracking algorithm}
\label{fig:coseg-fig3}
\end{figure}

\fbox{
\begin{minipage}{1.0\textwidth}
    {\bf Algorithm \texttt{cosegmentation-track}}
      \begin{itemize}
      \item \emph{Input:} Sequence of images $F_1,\dots,F_N$; pruning
        parameters $\theta_{\min},\theta_{\max}$, single-node cutoff
        $\sigma$.
      \item \emph{Output:} Sequence of segmented images $S_1',\dots,S_N'$.
      \end{itemize}
    \begin{enumerate}
    \item Compute component tree $T_i$ for all $i\in[1:N]$
    \item Prune $T_i$ to obtain $T_i'$ using
      $\theta_{\min},\theta_{\max},\sigma$.
    \item For each $i\in[1:N-1]$, compute
      $A_i=\mathrm{treeassign}(T_i,T_{i+1})$.
    \item Use $A_i$ and $A_{i+1}$ to obtain two segmentations of image
      $F_i$; compute consensus segmentation $S_i$ from these two.
    \item For each $i\in [1:N-1]$, compute a maximum-weighted bipartite
      matching between the segments in $S_i$ and $S_{i+1}$.
    \item Assign random color to each segment in $S_1$ to obtain $S_1'$. In $S_{i+1}'$,
      assign the same color to the segment as the one matched in $S_i$
    \end{enumerate}
  \end{minipage}
}

\subsection{Turning tree assignments into cosegments}
Assume $A=\{(a_1,b_1),\ldots,(a_k,b_k)\}$ is a tree assignment between the component trees of two images $I$ and $J$. Then the areas $\beta(a_1),\ldots,\beta(a_k)$, where $\beta(a)$ is defined in Section \ref{sec:alpha-beta-area},  refer to pairwise disjoint segments in $I$, and can hence be considered as a segmentation of $I$. Correspondingly, the areas $\beta(b_1),\ldots,\beta(b_k)$ induce a segmentation of $J$; note that the segments $\beta(a_i)$ and $\beta(b_i)$ necessarily overlap, as they require a non-zero weight to be included in an optimal tree-assignment.

\subsection{Consensus segmentation and bipartite matching}
When determining a segmentation of frame $i$, we are confronted with two competing options; one segmentation $P_i$ resulting from the cosegmentation of frames $i-1$ and $i$, the other one, $Q_i$, from the cosegmentation of frames $i$ and $i+1$. We resolve this by computing a consensus segmentation. We generally use $P_i$ as the starting point of a consensus segmentation. Any segment in $Q_i$ that does not overlap with any segment $P_i$ is supplement to $P_i$ to obtain the consensus segmentation $P_i′$. This ensures that cells entering the scene in frame $i$ can be identified in frame $i$ (rather than frame $i+1$).

\subsection{Filtering results}
As for most segmentation and tracking approaches, the results obtained from the steps described above has a tendency toward overdetection, i.e. detecting segments that result from image noise rather than cells. To filter out those segments, we utilize life span filtering, i.e. we filter out all cells whose identity can be traced across less than a certain minimum number of frames. This cell filter, along with several other ways to eliminate cells with unsuitable size or volume features, follows corresponding features of the Celltrack software \cite{sacan2008celltrack} for 2D cell tracking; for ct3d, they are implemented in the graphical user interface of the at3d tool shown in Figure \ref{fig:coseg-fig4}.

\begin{figure}[htbp]
\centering
\includegraphics[width=.5\textwidth]{images/coseg_fig4}
\caption[Screenshot of at3d]{Screenshot of at3d, which is designed for exploring cell tracking results and extracting motility parameters. It also supports features for correcting overdetection and oversegmentation. Cells can be selected and removed either individually or by filtering based on different criteria such as size or life span.}
\label{fig:coseg-fig4}
\end{figure}

\section{Implementation}
We implemented component trees, tree-assignments and the complete cell tracking algorithm, in C++ using lp\_solve\footnote{http://lpsolve.sourceforge.net/5.5/} for solving both the tree assignment and the weighted bipartite matching (integer) linear programs, all of which is compiled in the ct3d command line tool. Cell tracking results can be further explored using at3d, which allows the user to select and extract specific cells identified by the cell tracking procedure, and derive their motility parameters such as velocity and deformation. The at3d tool is implemented using the qt framework for graphical user interfaces. Input and output of image series is designed to be compatible with other visualization software, most notably v3d \cite{peng2010v3d} for producing rendered visualizations of the output.
\section{Experimental materials and methods}

3D time-lapse two-photon microscopy imaging of zebrafish microglia was
performed as follows: \emph{Zebrafish preparation}. Zebrafish
\emph{Tg(ApoE:egfp)}, in which microglia express EGFP \cite{Peri:08},
were maintained in the National Zebrafish Resources of China (NZRC,
Shanghai, China) with an automatic fish housing system (ESEN, Beijing,
China) at 28$^{\circ}$C. Embryos were raised at 28.5$^{\circ}$C under
a 14/10 hour light-dark cycle in 10\% Hank’s solution, which consists
of (in mM) 140 NaCl, 5.4 KCl, 0.25 Na2HPO$_4$, 0.44 KH$_2$PO$_4$, 1.3
CaCl$_2$, 1.0 MgSO$_4$ and 4.2 NaHCO$_3$ (pH 7.2). They were staged as
previously described \cite{Kimmel:95}. Zebrafish handling procedures
were approved by Institute of Neuroscience, Shanghai Institutes for
Biological Sciences, Chinese Academy of Sciences.

\emph{Time-lapse imaging.} For \textit{in vivo} imaging, zebrafish
larvae at 5-7 days postfertilization (dpf) were first anesthetized
with Hank’s solution containing 0.02\% tricaine methanesulfonate
(MS222) and embedded in 1.5\% low-melting point agarose. Time-lapse
images of microglia were captured, via a 40X water objective mounted
on a two-photon microscope with 900 nm (Prairie) or a Nikon A1R
confocal microscope with 488 nm. Z-stack images, which covered the
whole area of microglia in the optic tectum, were collected every 2
min at a section thickness of 1 $\mu$m. Each frame (512$\times$512
pixels, 14 to 34 Z-stack images) was averaged 4
times.

\section{Results}
\label{sect:results}

We evaluated our algorithm on two types of data. First,
   we applied it to an \textit{in vivo} time-lapse sequence of 3D
   two-photon images of zebrafish midbrain, displaying the motility of
   microglia; second, we applied \texttt{ct3d} to synthetically
   generated data for quantifying the accuracy of our cell tracking
   results.

  Evalutation on \textit{in vivo} data was accomplished by comparison
  with manually annotated trajectories of specific microglia in three
  data sets. Note that manual annotation is limited to trajectories,
  whereas boundaries of the cell volumes are almost impossible to
  obtain in 3D, beside systematic problems with manual annotations
  \cite{Huth:10}. Hence, we additionally created synthetic ground
  truth data to further evaluate the performance of our method. We
  followed the procedure used in \cite{dufour2005segmenting}, generating (noise
  perturbed) elliptical objects of average intensity $I_o$ above
  Poisson distributed background noise of intensity $I_b$. In addition
  to the procedure from \cite{dufour2005segmenting}, we created perturbed images
  with different types of background inhomogeneities, as shown in
  Figure \ref{fig:coseg-fig5}: in a second set of data, we
  introduced a multiplicative \emph{vignetting} effect to the data,
  following the vignetting model by \cite{kang2000can} under different
  focal lengths $f$ and off-axis illumination parameters $\alpha$. In
  a third set of data, we introduced an additive linear gradient along
  the $x$-axis of different slopes $\beta$. Each time series consists
  of 20 time frames, each of size $200\times 200\times 40$ pixels. On
  these data, we ran \texttt{ct3d} with size cutoffs
  $\theta_{\min}=500$ and $\theta_{\max}=5000$, and a single-node
  cutoff of $200$ pixels. In the resulting sequences, all cells whose
  identity could be traced through the complete sequence were kept,
  while all other cells were discarded using the \texttt{at3d} tool.


\subsection{Evaluation on synthetic data}

Our results on the synthetically generated image sequences are
summarized in Table \ref{tab:coseg-tab1} and indicate that
\texttt{ct3d} is highly robust against different types and intensities
of background inhomogeneities. The results suggest that \texttt{ct3d}
has a tendency to identify components slightly (about 10\%) larger
than the actual objects. This is a natural consequence of pruning the
component trees, where the vertex that would perfectly represent an
object is unlikely to be part of the pruned tree. This effect can be
reduced by smaller choice for the single-node cutoff parameter
$\sigma$ at the cost of higher computation time.

\begin{table*}
  \caption{Tracking results for synthetic data. }
  \centering
	\footnotesize
  \begin{threeparttable}
  \begin{tabular}{|c|ccc|rrr|rrr|}
\hline
&
\multicolumn{3}{|c}{\emph{Data Set}} &
\multicolumn{3}{|c}{\emph{\texttt{ct3d} Tracking Result}} &
\multicolumn{3}{|c|}{\emph{Chan-Vese Result}} \\\hline
&$I_o$ & $I_b$ &\emph{inhomogeneity}&\#cells&voxel recall& error rate            &\#cells         &voxel recall  & error rate\\\hline
\multirow{3}{*}{\begin{sideways}\emph{none\,\,\,\,\,\,}\end{sideways}}
& 2    & 1     & -                  & 3.00  & 99.94\%    &        9.86\%         &3.00	& 92.94\%	& 7.07\%\\	
& 3    & 1     & -                  & 3.95  & 100.00\%   &       22.93\%         &3.00	& 99.25\%	& 0.75\%\\	  
& 3    & 2     & -                  & 3.00  & 100.00\%   &       11.48\%         &3.00	& 90.25\%	& 9.75\%\\	
& 6    & 1     & -                  & 3.00  & 100.00\%   &        9.74\%         &3.00	& 99.70\%	& 0.30\%\\	
& 10   & 1     & -                  & 3.00  & 99.11\%    &       10.05\%         &3.00	& 99.93\%	& 0.07\%\\	
\hline\multirow{3}{*}{\begin{sideways}\emph{linear\,\,\,\,\,}\end{sideways}}      	  		  		
& 3    & 1     & $\beta=2$          & 3.00  & 100.00\%   &       10.62\%         &2.80	& 80.26\%	& 19.95\%\\	
& 3    & 1     & $\beta=3$          & 3.00  & 100.00\%   &        4.51\%         &2.10	& 62.97\%	& 39.00\%\\	
& 3    & 2     & $\beta=3$          & 3.00  & 99.97\%    &        6.83\%         &1.95	& 58.27\%	& 42.76\%\\	
& 6    & 1     & $\beta=3$          & 3.00  & 100.00\%   &        1.27\%         &2.20	& 66.54\%	& 41.26\%\\	
\hline\multirow{3}{*}{\begin{sideways}\emph{vignetting\,\,}\end{sideways}}        	  		  		
& 2    & 1     & $f=100,\alpha=5e-3$& 3.00  & 98.42\%    &        8.13\%         &0.85	& 22.89\%	& 77.25\%\\	
& 2    & 1     & $f=200,\alpha=1e-3$& 3.00  & 99.96\%    &        9.08\%         &1.90	& 56.56\%	& 43.60\%\\	
& 2    & 1     & $f=200,\alpha=5e-3$& 3.00  & 98.45\%    &        6.80\%         &0.60	& 15.76\%	& 84.31\%\\	
& 3    & 1     & $f=200,\alpha=1e-3$& 3.00  &100.00\%    &       14.25\%         &3.00	& 93.70\%	& 6.33\%\\	
\hline
  \end{tabular}
\begin{tablenotes}
\footnotesize
\item To assess the quality
    of tracking results, we derived the number of identified cells, the
    \emph{voxel recall}, i.e., what percentage of all ground-truth
    object voxels was recovered in the tracking result, as well as the
    \emph{error rate} defined as the ratio between the cardinality of 
    the symmetric difference between tracking result and ground truth 
    and the total number of voxels in the ground truth dataset. 
\item \emph{Left columns:} Each
    dataset displayed three cells with different intensities $I_o$ above
    different levels of noise $I_b$, displaying different types of
    background inhomogeneities (see text). 
\item \emph{Middle Columns:} The
    three cells were correctly recovered under all settings by
    \texttt{ct3d}. 
\item \emph{Right Columns:} Segmentation using the active 
    contour approach by \cite{chan2001active} is highly reliable in the absence of 
    background inhomogeneity.
    With increasing level of inhomogeneity, voxel 
    recall decreases, while the error rate increases.
\end{tablenotes}
  \end{threeparttable}
  \label{tab:coseg-tab1}
\end{table*}

Running times varied between roughly 4 and 6 minutes, with an average
of $303.42$ seconds, for completely tracking one dataset. The majority
of running time was spent on constructing the component trees and
computing the overlap weights ($14.23$ seconds on average per time
frame), whereas each tree assignment required less than one second on
average; the pruned component trees typically comprised a few dozens
of vertices.

As a reference algorithm to compare the performance of \texttt{ct3d}
against, we computed a segmentation of each time frame using the
active contour approach by \cite{chan2001active}\footnote{Results were
  computed for parameters $\mu=1$, $\nu=.7$, $\lambda_1=1$,
  $\lambda_2=2$, $\Delta t=.8$ running 100 iterations.}, which is a
well-established and state-of-the-art representative of the large
family of level set methods. As shown in Table
\ref{tab:coseg-tab1}, this method works highly accurate in the
absence of background inhomogeneity while getting less reliable with
increasing levels of background inhomogeneity, as can be expected due
to the involvement of a global background model.

\begin{figure}[htbp]
\centering
\includegraphics[width=1.0\textwidth]{images/coseg_fig5}
\caption[Sections of synthetically generated images]{Sections of synthetically generated images—Left: 2D section (top) of an image perturbed background vignetting following the Kang–Weiss model\cite{kang2000can}. The dashed box indicates the position of the 1D section displayed in the lower part. Right: same setting with the background perturbed by an additive linear gradient. In both instances, ct3d yields reliable tracking results (see Table \ref{tab:coseg-tab1}).}
\label{fig:coseg-fig5}
\end{figure}

\begin{figure}[htbp]
\centering
\includegraphics[width=0.6\textwidth]{images/coseg_fig6}
\caption[Cosegmentation results comparison with other methods]{Top: visualization of trajectories obtained by manual annotation (blue lines) with trajectories obtained using ct3d (orange lines) for microglia in activated state, see (a) and (b), as well as resting state, see (c) and (d). Bottom: quantitative comparison of trajectories obtained by ct3d and the active contour approach from Chan and Vese \cite{chan2001active}. In general, ct3d could identify the annotated cells in all time frames. The root mean square distance to the annotated trajectory measures a fraction of the diameter of the annotated cell (columns cell $\Delta$ and RMS), while the Chan–Vese algorithm missed varying numbers of cells or failed completely (last column). For the Chan–Vese results, a parameter set was optimized for dataset (a). This parameter set ($\mu$=1, $\nu$=.5, $\lambda_1$=.2, $\lambda_2$=5) was also applied to datasets (b) to (d). While for dataset (a), the result is comparable to ct3d, the annotated cell was identified in only 29 out of 49 time frames in (b). In datasets (c) and (d), the annotated cell could not be identified at all using these parameters.}
\label{fig:coseg-fig6}
\end{figure}
\subsection{Tracking microglia \textit{in vivo}}

Figure \ref{fig:coseg-fig6} shows a result obtained from
our tracking algorithm on a time series of microglia images measured
as described above. We reduced resolution by half, so that the
resulting width and height varied between 146 and 250 pixels, while
the depth ranged between 14 and 66 layers for each time frame; each
time series comprised 30 to 80 time frames. Gray scale resolution was
reduced from 16 bit to 8 bit. We applied \texttt{ct3d} using
parameters $\theta_{\min}=200$, $\theta_{\max}=10,000$ and a
single-node cutoff of $200$ pixels; the resulting pruned component
trees contained 69 vertices on average, ranging between 40 and 168
vertices. Running times varied between roughly 2 to 10 minutes, with
471 seconds on average. 

Under the given experimental protocol, the phenomenon of
\emph{overdetection}, i.e. the recognition of segments that are not
microglia, is inevitable. This is due to the limited specificity of
the \emph{apoE-GFP} gene, which is also expressed in cells other than
microglia in the surrounding tissue, often at comparably high levels
as in microglia. Yet, \texttt{ct3d} identifies microglia as segments
that can be visually distinguished from non-microglia segments by a
human observer due to their characteristic shape or motion
patterns. The graphical user interface of the \texttt{at3d} tool (see
Figure \ref{fig:coseg-fig4}) allows to manually eliminate false positive
cells from tracking results by different filtering and visual
selection functions similar to those provided by \texttt{Celltrack}
for two dimensional time lapse sequences. The \texttt{at3d} tool also
allows to manually correct for the occasionally observed events of
oversegmentation, i.e., one microglia being recognized as two segments.

We used \texttt{at3d} to eliminate non-microglia from all six datasets
and correct oversegmentation in individual frames. We were able to
reconstruct trajectories of all relevant microglia that can be
visually identified in the original datasets; only one dataset was
affected by a sudden ``frameshift'', i.e., the sample changing its
distance to the camera in the $z$ direction, leading to interrupted
trajectories at the corresponding time frame. In few other cases, the
trajectory of a microglia was interrupted in individual frames, which
we could correct using \texttt{at3d}.

In order to quantitatively evaluate the quality of \texttt{ct3d}
results on \textit{in-vivo} microglia data, we compared \texttt{ct3d}
trajectories with manual annotations. To this end, we selected four
microglia from four different datasets and annotated their
trajectories manually using the 3D polyline markup feature of the
\texttt{v3d} software. The root mean square distance between the
\texttt{ct3d} and the manual trajectories turned out to be 2.9 voxels,
5.5 voxels, 2.6 voxels, and 5.4 voxels, respectively, in the four
different datasets. These deviations can be considered relatively
small in relation to the cell diameters, which were measured as 22.8,
22, 19.6, and 44.3 voxels, respectively.

We also applied the active contour approach from \cite{chan2001active} to
the datasets. Identifying the the four microglia in our annotated
evaluation dataset required intensive tuning of the four major
parameters. In fact, we could not identify a single set
  of parameters that works across all datasets. A parameter set tuned
  for a specific dataset worked comparably well on the respective
  dataset, but performed significantly worse or yielded no result on
  other datasets, see Fig. \ref{fig:coseg-fig6}.

\section{Discussion}

We have presented a novel approach to tracking cells in three
dimensional time lapse microscopy image sequences, based on the
concepts of component trees and cosegmentation. We demonstrate that
this approach is robust against the numerous challenges imposed by
images measured in an \textit{in vivo} environment, and allows to
identify microglia and their motion patterns in zebrafish neural
tissue when combined with the \texttt{at3d} annotation tool. In a
quantitative evaluation, we show that our approach is robust against
different types of background inhomogeneities. This suggests that
\texttt{ct3d} and \texttt{at3d} are potentially useful for \textit{in
  vivo} imaging studies investigating other aspects than just
microglia motility. In its current formulation, our cosegmentation
approach relies on the assumption that the area occupied by an object
overlaps between two consecutive time points. While this may not be
satisfied in all cell tracking problems (e.g. when tracking
centrosomes \cite{Jaensch:10}), it is a reasonable assumption for
many immunoimaging related studies.

To the best of our knowledge, our approach is the first that can
identify and track microglia in live cell imaging time series. In most
cases, obtaining reliable trajectories still requires manual
post-processing of the output. The most notorious difficulties
certainly are the complex morphology -- their deformation patterns,
irregular shapes, and interaction with the surrounding -- as well as
the unspecificity of the fluorescent markers available. In this light,
our approach constitutes significant progress in the sense that it has
sufficient sensitivity to separate microglia form their
surrounding. Yet, a fully automated approach remains a major and
certainly non-trivial challenge. A first step in this direction might
be the combination with level-set based approaches as utilized for 2D
cell tracking by \cite{Nath:06} that might yield more accurate cell
boundaries in some cases. Yet, \texttt{ct3d} promises to be a key tool
for further studying open questions regarding microglia, such as to
determine if and how glia and microglia share the task of finding and
removing apoptotic neurons from the vertebrate brain \cite{Peri:08}. 

Beside the direct relevance for \textit{in vivo} time lapse
microscopy, our study indicates that our morphological approach to
cosegmentation is both practical and of relevance in
bioimaging. Consequentially, it appears a natural approach to apply
cosegmentation to protein colocalization studies, which have attracted
considerable attention in recent years following the availability of
two- or multi-label fluorescence microscopy \cite{Zinchuk:08}.

Another major experience that can be drawn from our work is the obvious
potential of component trees and the closely connected theory of
component filters in bioimaging. While component filters are well
known to leave relevant gradients unchanged, recent work such as the
results by \cite{Coupier:05} allow to assign a
statistical significance to components observed in an image. Such
concepts might be particularly useful when combining component trees
with cosegmentation for judging the relevance of colocalized segments
observed when comparing two component trees.

\section{Contribution of this thesis}
In this thesis, we build a cosegmentation based pipeline for 3D time-lapse microglia cell tracking.
The novel cosegmentation methods accomplish cell segmentation and cell association simutaneously through 
component tree assignment. We evalute our method on synthetically generated data, demonstrating that our algorithm is robust even in the presence of different types of inhomogeneous background noise. Our algorithm is implemented in the ct3d package, which is available under http://www.picb.ac.cn/patterns/Software/ct3d.
 % chapter4: cell tracking methods
\part{Neuron Tracing}
\chapter{Fastmarching method}
\section{Algorithm flow}
\section{Different Applications}
\subsection{Shortest path between any shapes}
\subsection{Gray-weighted distance transfrom}
\subsection{Advaned voronoi diagram}

\chapter{Hierarchical pruning based automatic neuron tracing methods} \label{chpt:auto-nt}
In this chapter we focus on neuron tracing for 3D light-microscopic images (often confocal or multi-photon laser scanning microscopic images). Taking a 3D gray-scale image as the input, a neuron tracing method produces the digital representation of the morphology of the neuron(s) in this image. Tracing multiple neurons that potentially overlap in an image has been found to be fundamentally ill-posed and might be ultimately tackled using biological tissue labeling methods, such as dBrainbow \cite{hampel2011drosophila}. In the latter case, the problem reduces to tracing of a group of single-channel images, each of which contains a single neuron. Therefore, here we discuss only how to reconstruct a single neuron’s morphology from an image. 
\section{Overview}
\subsection{Local/Global neuron tracing methods}
\subsection{APP1 method}
\section{APP2 method}
\subsection{Overview of APP2}
In many previous studies, the 3D reconstructed morphology of a neuron is described using a tree graph $G$.  $G$ has a root node that corresponds to the seed location for reconstruction, which in many cases also corresponds to the soma of a neuron. $G$ may also contain many leaf nodes, branching nodes and other internodes.

An important idea in the recent all-path-pruning (APP) method \cite{peng2011automatic} is to first generate an over-reconstruction from the image to capture all possible signal/pixels of a neuron, and then uses an optimal pruning procedure to remove the majority of spurs in this over-reconstruction to produce a final succinct representation of the neuron, with a maximum coverage of all neuron signal. The pruning process starts from leaf nodes of the over-reconstruction. A "coverage" test is iteratively conducted to check whether or not any of them have been "covered" (i.e. has significant signal overlap) by other nodes. The nodes that are covered by others will be removed; otherwise they will be kept. A similar process is also applied to all internodes. The entire procedure is repeated until a most succinct representation that maximizes the signal coverage has been produced. While the termini-first-search approach in APP is effective, it needs multiple iterations, which could still be time-consuming even the algorithm itself has linear-time complexity to the number of reconstruction nodes. In addition, APP does not consider how to best preprocess an input image to optimize the tracing result. 

APP2 follows the basic framework of APP. However, it enhances the key components of the original method. In short, APP2 consists of three steps in Fig.\ref{fig:autont-fig1}b, c, and d: (1) gray-weighted distance transform (Section \ref{sec:gwdt}); (2) construction of an initial, over-reconstruction of the traced neuron (Section \ref{sec:init-nt}); and (3) pruning of the over-reconstruction in hierarchical order (Section \ref{sec:nt-hp}). The detail of APP2 is described as follows.

\begin{figure}[htbp]
\centering
\includegraphics[width=1.0\textwidth]{images/autont_fig1}
\caption[Schematic illustration of APP2 neuron tracing method]{Schematic illustration of APP2 neuron tracing method. GWDT: gray-weighted distance transform. In c and d, the reconstructions are color-rendered and overlaid on top of the image data for better visualization. Raw image: Courtesy of Chiang lab.}
\label{fig:autont-fig1}
\end{figure}
\subsection{GWDT: Gray-Weighted Distance transfrom} \label{sec:gwdt}
To enhance the quality of an initial neuron reconstruction, in APP2 we apply the distance transform (DT) to the input image. In case of an image region that have relatively "flat" intensity, DT is able to create a gradient of image intensity: close to the center of this region the image intensity is large, and close to the boundary the intensity is small. We call this ICDB principle, which stands for "\emph{i}ncrease the intensity in the \emph{c}enter and \emph{d}ecrease the intensity near \emph{b}oundary". It would help to build a high quality initial reconstruction by forcing the shortest path to go through the skeleton of the neuron.

However, the conventional distance transform is only applicable to a binary image that is produced by thresholding a gray-scale image. An unsuitable threshold may under or over segment the image. Here we propose a gray-weighted distance transform method for gray-scale image (GWSDT) directly. In the conventional distance transform, the distance value for each image pixel is defined as the minimal Euclidean distance to background pixels. In GWSDT, the distance value for each pixel is defined as the sum of image pixels’ intensity along the shortest path to background. This is intrinsically similar to the gray weighted distance transform that has been previously defined by Rutovitz (1968)\cite{rutovitz1968data} and its variants. Most of the previous work and implementations (such as the recently released Matlab toolbox function) were limited to 2D cases, while our method and implementation are general for N-dimensional data ($N = 2,3,\ldots$). In the following we describe our fast implementation within FM framework.   

The distance value defined in GWSDT fits the ICDB principle. To use GWSDT, we often use a very low threshold value (e.g. the average intensity of an entire image). Any image pixels that have intensity value no greater than this threshold are called "background pixels". We first set all image background pixels as "seeds", then compute the distances from these seed pixels to other pixels. This process is similar to region growing, and thus is formulated within the FM framework in Chapter \ref{chpt:fm}. Here, the edge distance between consecutive image pixel vertices is defined as,
\begin{equation}
e(x,y)=|(|x-y|)|.I(y)
\end{equation}
Where $x$ is source vertex, $y$ is target vertex, $I(y)$ is the intensity of image pixel $y$. Let $d(x)$ denote the distance value of $x$. In the initialize step, we set
\begin{equation}
d(x)= \left\{
\begin{array}{rl}
I(x)  & x \in {background} \\
\infty &    x \notin {background } 
\end{array}
\right.
\end{equation}
We will set all background pixels as seeds. The neighbor pixels of all seeds will be set as the initial \emph{TRIAL} vertices and are pushed into the priority queue. In the growing step we will apply the formula,
\begin{equation}
d(x)=\min⁡{d(y)+ I(x)},y \in \{\mbox{neighbors of }x\}
\end{equation}
to refresh the distance value from background to skeleton center.

\textbf{Automated Soma detection}: We find that GWSDT method provides a way to detect the soma position of a neuron. Normally the soma has the maximum distance-transformed value (e.g. see results in \ref{}).

\subsection{Initial neuron reconstruction}\label{sec:init-nt}
In APP, the initial neuron reconstruction is essentially produced via finding the single-source (often from the soma) shortest path to all remaining foreground image pixels. APP uses Dijkstra’s algorithm, which needs to first build a graph of all foreground pixels and then find the shortest path from each pixel to the seed. For very large 3D image stacks, this approach may need a large amount of computer memory to hold the graph. In APP2, we present a new method to use FM to construct the initial reconstruction (Fig. \ref{fig:autont-fig2}), without the need to create such a large graph. 

In our implementation, we add a parental map par on the FM as described in Chapter \ref{chpt:fm} to generate the shortest path tree from a single source s.  Initially, the parent of each image pixel $x$ is set to be itself, i.e. $par(x)=x$. Then, for each neighbor pixel $y$ of $s$, we set them to have label "\emph{TRIAL}", and at the same time $par(y)=s$. In the recursive step, for the minimum pixel $x$ and each of its neighbor $y$, we set
\begin{equation}
par(y) = \left\{
\begin{array}{cl}
x & \mbox{if }y\mbox{ is }FAR \\
x & \mbox{if }d(x)+ e(x,y)< d(y)
\end{array}
\right.
\end{equation}

The edge distance e(x,y) is defined as the geodesic distance,
\begin{equation}
e(x,y)=\parallel x-y\parallel\cdot((g_I (x)+ g_I (y))/2)
\end{equation}
where the first item is Euclidean distance of the two pixels, and $g_I (.)$ in the second item is defined in the same form of APP \cite{peng2011automatic}, where $λ_I$ is a coefficient (set as 10 throughout our tests). 
\begin{equation}
g_I (x)=exp⁡(\lambda_I (1-I(x)/I_{max} )^2 )
\end{equation}
When FM has finished, we can build the initial reconstruction from the parental map.

In addition to the small working space needed, another useful property of FM for generating the initial reconstruction is that it can be stopped easily as needed. We consider two methods in APP2. First, the recursive step will stop when any background pixel becomes \emph{ALIVE}. This method prevents the marching process growing to any irrelevant area. Second, a user can optionally choose to specify some locations in advance (e.g. some special termini of a neuron) as additional priors; when all of these special locations have been labeled as \emph{ALIVE}, FM stops. The second method forces FM to reach these specified locations. Both methods are used in practice to generate complete initial reconstructions that cover signals as much as possible and thus make it easier to trace broken pieces or gaps in images. 

\begin{figure}[htbp]
\centering
\includegraphics[width=1.0\textwidth]{images/autont_fig2}
\caption[Initial reconstruction based on GWDT and comparison results with or without GWDT]{Initial reconstruction based on GWDT (image: DIADEM OP1) and comparison results with or without GWDT. (a) The main GWDT-based skeleton (the major mid-curves) of the tracing implies a good tracing.  Sphere: the seed location. (b) left: out-of-center problem without GWDT preprocessing; right:  main structure goes to the neuron center with GWDT preprocessing . (c) top: parallel-path problem without GWDT preprocessing; bottom: parallel paths disappear after GWDT preprocessing.}
\label{fig:autont-fig2}
\end{figure}
\subsection{Hierarchical pruning}\label{sec:nt-hp}
Fig. \ref{fig:autont-fig2}a displays an example of an initial reconstruction of neuron. It is essentially a tree graph with a number of spurs. The next step is to find the main skeleton of the tree by removing the unnecessary or redundant spurs. Here we propose a hierarchical pruning method that contains two steps: a hierarchical segments construction step and a recursive pruning step. 
\subsubsection{Hierarchical segments construction}
For simplicity, we call the initial reconstruction a "tree" in this section. We define a segment in the tree as a path connecting two branching nodes in the tree. In the hierarchical segments construction step, we order all segments in the tree from most important to the least important and thus generate a hierarchy of them. The “importance” of a segment is defined based on its length. The longer a segment, the more important it is. Obviously, there is no overlap between any pair of segments. To get the importance scores, we first find the longest path from the most distal leaf node to the source node (seed). We then delete this segment from the tree and find the second longest segment from the remaining parts in the original tree. We iterate this process until all segments have been sorted. 

We further improve the efficiency of the algorithm as follows. We observe that in our decomposition of a tree, each hierarchical segment starts from a leaf node. In addition, there is a one-to-one mapping relationship between each hierarchical segment to a leaf node. Therefore, in a refined algorithm, we construct the hierarchical segments in a bottom up order. First, we detect the segment from each leaf node to its nearest branch node (excluding the branch node). Each branch node connects to at least two such segments. Then we merge the branch node to the longest segment (called "joint-segment"). Next, the other segments originally connects to the branch node are set as child segments to the joint segment. We iterate this joint segment merging process until the seed node is reached. 
\subsubsection{Recursive pruning}
In the pruning step, from the pool of undeleted hierarchical segments we choose one segment at a time following the importance-score in decreasing order. Then, we extend the signal-coverage idea in APP to coverage ratio of this segment (see below). If the coverage ratio is larger than a threshold value (normally 75\%), we delete this segment and all its child-segments. Otherwise, we keep this chosen segment and mask the coverage area of the segment. We iterate this process until no segment can be removed.

The coverage area of a segment is defined as the merged coverage area of all nodes in the segment. The coverage area of a node is defined as the sphere area centered at the node, with an estimated radius of the node, which is computed using the method described in Peng et al \cite{peng2010v3d,peng2010automatic}.

The coverage ratio of a segment is defined as the ratio of the number of all masked nodes with respect to the total number of nodes in the segment. Further, we consider image pixels with different intensities should have different weights. Thus in our scheme, we use the image-pixel-intensity weighted coverage ratio, which is defined as the sum of intensity of all masked nodes divided by that of all nodes in the segment.
\section{Experimental results}
\subsection{Results between GWDT and Non-GWDT}
We compared the new GWDT step in APP2 to those generated without GWDT (see Fig. \ref{fig:autont-fig2}). It is clear that when GWDT is not used (left of Fig. \ref{fig:autont-fig2}b), the detected skeleton of the neuron is often skewed to one side of the shape. This is effectively avoided when GWDT is used (right of Fig. \ref{fig:autont-fig2}b); the respective skeleton best covers the neurite signal in a balanced way.

Fig. \ref{fig:autont-fig2}c shows that for complex branching patterns, the shortest path algorithm could easily produce "parallel paths" when GWDT is not used (top of Fig. \ref{fig:autont-fig2}c). This problem is also clearly overcome after GWDT is applied to the image (Fig. \ref{fig:autont-fig2}c bottom). For this test image, the overall morphology of the GWDT-based result (Fig. \ref{fig:autont-fig2}c bottom) appears to be more reasonable than the non-GWDT result.

Of course, when GWDT is not invoked, APP2 can run faster (Table \ref{tab:autont-tab1}), although the accuracy might be compromised in a way similar to Fig. \ref{fig:autont-fig2}.

\begin{table} \label{tab:autont-tab1}
  \caption{The numbers of tree-segments pruned by APP2 and APP when sequentially applying APP2 or APP methods in pruning}
\begin{center}
 \begin{tabular}{ccccc}
    \hline
	\multirow{2}{*}{Neuron image} &\multicolumn{2}{c}{First APP2 then APP} & \multicolumn{2}{c}{First APP then APP2}\\ \cline{2-5}
	 &APP2 & APP & APP & APP2\\ \hline
	OP1 fly & 9421& 0 & 9180& 246 \\ \hline
Chiang fly	& 42069	& 3 &	34185 &	6529\\ \hline
Dragonfly C147 &	411	& 0	& 252 &	159\\ \hline
Dragonfly C150 &	10082 &	1 &	6709 &	3410\\ \hline
Dragonfly C152 &	662	& 0 & 	588 &	74\\ \hline
Dragonfly C154 &	969	& 0	& 598	& 368\\ \hline
Dragonfly C157 &	6084 &	3	& 2350 &	3731\\ \hline
Dragonfly C158 &	60331 &	25 &	44216 &	16499\\ \hline
Dragonfly C159 &	395	& 0	& 266 &	129\\ \hline
Dragonfly C160 &	7787 &	0	& 2639 &	5118\\ \hline
Dragonfly C161 &	45409 &	22 &	31098 &	14606\\ \hline
Dragonfly C162 &	328	& 1	& 293 &	38\\ \hline
Dragonfly C165 &	2854 &	1 & 	744 &	2108\\ \hline
Dragonfly C168 &	2537 &	0 &	1777 &	753\\ \hline
Dragonfly C169 &	981	& 0	& 720 &	253\\ \hline
Dragonfly C171 &	2260 &	0 &	1509 &	750\\ \hline
Dragonfly C173 &	2189 &	0 &	1662 &	518\\ \hline
Dragonfly C175 &	4830 &	0 &	3239 &	1555\\ \hline
Dragonfly C179 &	1905 &	0 &	1378 &	518\\ \hline
Dragonfly C180 &	456	& 0	& 350 &	106\\ \hline
Dragonfly C183 &	195	& 0	& 145 &	50\\ \hline
Dragonfly C188 &	2785 &	0 &	2401 &	381\\ \hline
Dragonfly C189 &	383	& 0	& 331 &	55\\ \hline
Dragonfly C190 &	39	& 0	& 25 &	14\\ \hline
Dragonfly C192 &	1048 &	0 &	832 &	215\\ \hline
Dragonfly C193 &	624	& 0	& 464 &	163\\ \hline
Dragonfly C194 &	1243 &	0 &	815 &	426\\ \hline
    \end{tabular}
\end{center}
\end{table}

\begin{table} \label{tab:autont-tab2}
\caption{Comparison of running time (seconds) of APP2 and APP on a few images. Testing is based on a MacPro laptop with 2.6GHz Intel Core i5}
\begin{center}
\begin{tabular}{cccc}
\hline
Image	& APP &	APP2 (w/ GWDT)	& APP2 (w/o GWDT) \\ \hline
Chiang fly & 149.7 &	28.9 &	12.6 \\ \hline
Dragonfly C154 &	31.6 &	6.9 &	1.6 \\ \hline
Dragonfly C168 &	48.8 &	7.9	& 1.9 \\ \hline
Dragonfly C171 &	44.2 &	10.2 &	2 \\ \hline
Dragonfly C190 &	32.6 &	7.6	& 1.3 \\ \hline
\end{tabular}
\end{center}
\end{table}

\subsection{Comparison with other methods}
We examined the robustness of APP2 by tracing images where signal were deleted \cite{peng2010automatic}. Three levels of signal deletion, 30\%, 60\% and 90\% were tested (Fig. \ref{fig:autont-fig3} and Fig. \ref{fig:autont-fig3-2}). We compared APP2 to several previous automated methods in the public domain, including APP \cite{peng2011automatic}, NeuronStudio \cite{rodriguez2008automated}, and SimpleTracing \cite{yang2013distance} , as well as the "ground truth" reconstruction, which was obtained by combining semi-automatic tracing and manual editing. To make a fair comparison, the reported results of competing methods correspond to the best possible parameters finetuned in our testing. 

We calculated several difference scores of the reconstructions produced by the automated methods and the "ground truth". These difference scores measure the "spatial distance" between a particular reconstruction and the ground truth, as well as the percentage of reconstruction elements that have significant, i.e. visible, distance to the nearest reconstruction elements in the "ground truth". These scores, as previously defined in Peng et al.\cite{peng2010v3d}, are called entire structure average (ESA), different structure average (DSA) and percent of different structure (PDS) for simplicity in this paper. 

Fig. \ref{fig:autont-fig3} shows that APP2 is able to achieve the lowest difference-scores among all methods. It actually consistently produced subpixel precision compared to other methods, such as NeuronStudio. For the DSA score, APP2 and APP are close to each other, but APP2 is better than APP in the ESA and PDS scores. 

\subsection{Improvement on APP}
Fig. \ref{fig:autont-fig3} indicates that the performance of APP2 is better, but still close to that of APP. While this observation is anticipated, it raises a natural question that how much APP2 will improve APP. In our design, the biggest difference between these two methods is how they prune the initial over-reconstructions. Therefore, we studied the ability of APP2 in pruning the initial reconstruction. We also compared the running speed of both methods.

We considered using either APP or APP2 to prune an initial re-construction, followed by using the other method (APP2 or APP) to check if there is any further redundancy in the reconstruction that can be removed. After applying this test to a number of complicated dragon fly neurons, we found that (Table \ref{tab:autont-tab1}) on average APP2 is able to prune most redundant segments in an initial reconstruction tree, leaving a very small portion of redundancy that can be detected by APP. On the other hand, when we apply APP first, APP2 is still able to remove many tree-segments. This shows the advantage of hierarchical pruning. In this sense, APP2 provides a complementary pruning scheme to the APP method.  

Table \ref{tab:autont-tab2} summarizes the running speed of APP2 versus APP for several testing images. It can be seen that APP2 is much faster on these images.

\subsection{Real applications in tracing different neuron images of different animals and from different labs}
We tested APP2 on a variety of real neuron data sets, including for instance the fruitfly neurons data used in DIADEM competition, the flycircuit.org database, and Janelia fly imagery database, as well as some very challenging dragonfly neuron data sets \cite{gonzalez2013eight} that have heavy noise. 
Fig. \ref{fig:autont-fig4} shows a few examples of tracing various data sets:
\begin{itemize}
\item[(a)]	For images that have very uneven image pixel intensity (e.g. Fig. 4c), APP2 is able to produce a complete reconstruction. 
\item[(b)]	For images that have fine branches (e.g. Fig. 4d), which are very easy to get missed by other tracing methods, APP2 is able to detect them reasonably well. 
\item[(c)]	For a neuron that may contain a big cell body (e.g. Fig. 4b), APP2 is able to detect the cell body automatically, and therefore reconstruct the entire morphology fully automatically.
\end{itemize}

\subsection{Large-scale reconstruction of single fruitfly neurons}
We applied APP2 to 678 3D 40x confocal images contributed by Chiang lab. Each image contains a single neuron labeled in a Drosophila brain. We ran both automatic soma detection and automatic neuron tracing in APP2. 

After manual proofreading of the tracing results against the original images, we found that for automatic soma detection, we had a success rate 96.6\% for this data set. The failure cases are mainly due to the insufficient pixel resolution in some of the images and thus poor separation in dense arborization areas. For automated neuron tracing, 629 (92.8\%) neurons were reconstructed reasonably well. The unacceptable tracing is mainly due to the poor image quality, i.e. broken pieces of neurite in the respective images. 

The 629 successfully traced neurons (Fig. \ref{fig:autont-fig4}e) make up one of the largest automatically traced Drosophila single neuron databases to date. These reconstructions will eventually be documented in publicly available neuron morphology databases such as NeuroMorpho.org. 

\begin{figure}[htbp]
\centering
\includegraphics[width=1.0\textwidth]{images/autont_fig3}
\caption[Comparison with different neuron tracing methods subject to the deletion noise]{Comparison with different neuron tracing methods subject to the deletion noise. The reconstructions are overlaid on top of the original images for better visualization. The fourth row corresponds to 90\% signal is deleted. Since it is so dark in this case, we set pixel intensity threshold to 1 so that we are able to compare all different methods. Right side subfigures: the three difference-scores compared to the “ground truth” reconstruction. NS: NeuronStudio. ST: SimpleTracing.}
\label{fig:autont-fig3}
\end{figure}
\begin{figure}[htbp]
\centering
\includegraphics[width=.7\textwidth]{images/autont_fig3_2}
\caption{The signal deletion result of APP2 on Fly OP1 data}
\label{fig:autont-fig3-2}
\end{figure}

\begin{figure}[htbp]
\centering
\includegraphics[width=1.0\textwidth]{images/autont_fig4}
\caption[Examples of tracing results on different neuron data sets contributed by different labs]{Examples of tracing results on different neuron data sets contributed by different labs. (a)-(d): tracing results of neurons of different model animals. (e) 3D reconstructed neurons of a large Drosophila neuron image data set with 678 neurons. Different colors indicate different neurons.}
\label{fig:autont-fig4}
\end{figure}

\section{Conclusion}
We present a new neuron tracing system that contains several novel algorithms based on fast marching and hierarchical pruning. We use the fast marching method to compute both the initial neuron reconstruction and the gray-weighted distance transform, and at the same time also improve the robustness of neuron tracing. Hierarchical pruning sorts the individual segments of an initial reconstruction tree in a hierarchical order, thus facilitates efficient removal of redundant segments in the reconstruction. We have compared our new method to various previous methods on a number of datasets and found a better performance of the new method in most cases. 

\chapter{Humman guided neuron tracing methods} \label{chpt:manual-nt}
\section{Difficulties for 3D drawing on 2D plane}
\section{Existing methods in Vaa3d}
\section{Fast marching based methods}

\bibliographystyle{plain}
\bibliography{thesis}

\end{document}
