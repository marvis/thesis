\section{Graph cuts method}
\subsection{Network flow theory}
Due to the theorem of Ford and Fulkerson\cite{ford1963flows}, this is equivalent to computing the maximum flow from the source to sink. There are many algorithms that solve this problem in polynomial time with small constants \cite{ahuja1993network,boykov2004experimental,goldberg1988new}
\begin{table}
  \caption{History of Maxflow Algorithms. The red rows use augmenting path method and the blue rows use push-relabel method. $n$ is the number of nodes, $m$ is the number of edges, and $U$ is the maximum edge weight.}
\begin{center}
 \begin{tabular}{| l | l | l |}
    \hline
    Year & Author(s) & Running time \\ \hline 
    1951 & Dantzig & $O(n^2mU)$ \\ \hline
	1955 & Ford \& Fulkerson & $O(m^2U)$ \\ \hline
	1970 & Dinitz & \color{red}$O(n^2m)$\\ \hline
	1972 & Edmons \& Karp & $O(m^2\log U)$\\ \hline
	1973 & Dinitz & \color{red}$O(nm\log U)$ \\ \hline
	1974 & Karzonov & \color{red}$O(n^3)$ \\ \hline
	1977 & Cherkassky & \color{red}$O(n^2m^{1/2}$ \\ \hline
	1980 & Galil \& Naamad & \color{red}$O(nm\log^2n)$ \\ \hline
	1983 & Sleator \& Tarjan & \color{red}$O(nm\log n)$ \\  \hline
	1986 & Goldberg \& Tarjan & \color{blue}$O(nm\log(n^2/m))$ \\ \hline
	1987 & Ahuja \& Orlin & \color{blue}$O(nm + n^2\log U)$ \\ \hline
	1987 & Ahuja et al. & \color{blue}$O(nm\log (n\sqrt{\log U}/m))$\\ \hline
	1989 & Cheriyan \& Hagerup & \color{blue}$O(nm + n^2 \log^2n)$\\ \hline
	1990 & Alon & \color{blue}$O(nm + n^{8/3}\log n)$\\ \hline
	1992 & King et al. & \color{blue}$O(nm + n^{2+\epsilon})$\\ \hline
	1993 & Philips \& Westbrook & \color{blue}$O(nm (\log_{m/n}n + \log^{2+\epsilon}n))$\\ \hline
	1994 & King et al. & \color{blue}$O(nm\log_{m/(n\log n)}n)$ \\ \hline
	\multirow{2}{*}{1997} & Goldberg \& Rao & \color{red}$O(m^{3/2}\log(n^2/m)\log U)$ \\
	 & & \color{red}$O(n^{2/3}m\log(n^2/m)\log U)$\\ \hline
    \end{tabular}
\end{center}
\end{table}
\subsubsection{Ford-Fulkerson algorithm}
\subsection{Markov Random Fields}

