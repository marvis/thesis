\chapter{Introduction}
\section{Bioimage informatics}
% techniques developed -> large dataset -> manually analysis unrealistic -> automatic/half automatic algorithm needed -> a new engineering area -> bioimage contribution (conference and challenge)
In the last decades, numerous bioimaging techniques were developed to monitor biological phenomena with unprecedented resolution, specificity, dimensionality, complexity, and scale. Some of the advanced biological imaging methods include confocal or two-photon laser scanning microscopy (LSM) \cite{pawley1995handbook}, scanning or transmission electron microscopy (EM) \cite{bozzola1999electron}, PALM \cite{betzig2006imaging}, STORM \cite{rust2006sub}, STED \cite{hell2003toward}. With these modern automated microscopes, a large number of 3D images can be collected very quickly, e.g. the light-sheet microscopes can result in 30 terabytes of data per day \cite{keller2008reconstruction}. The number of biological images (e.g. cellular, molecular and medical images) acquired in digital forms is growing rapidly. 

However, due to the lack of necessary analysis tools, such image has been analyzed manually. Manually annotation, e.g. segmentation of cells or tracing of neurons, is, however, slow, expensive, and often highly variable from person to person.

The deluge of complicated biological and biomedical images exposes the great demand for automatic or half-automatic image processing algorithms. As a natural extension of existing biomedical field, a new biological engineer field, that is, bioimage informatics is becoming more and more important in recent years when large scale image dataset appears to be very common.

This thesis, we make contributions to two problems in bioimage informatics, namely cell tracking and neuron tracing. In part II, We develop a novel algorithm for microglia cell tracking in 3D time-lapse microscopy \cite{Xiao:2011}, which is based on tree assignment \cite{xiao2011dynamic} between component trees of images. The algorithm finishes cell segmentation and cell association simultaneously, that is called cosegmentation. In part III, we develop methods for both human guided neuron tracing, which is based on fast marching algorithm for shortest path finding between view rays, and automatic neuron tracing \cite{xiao2013app2}, which is based on grayscale distance transform, distance tree construction, as well as hierarchical pruning process. We make lots of improvement on all the related techniques.

\section{Microglia Cell Tracking Analysis}
\subsection{Background}
Capturing the motility of cells using time-lapse microscopy has become an important approach to understanding processes such as the cell cycle \cite{Harder:09}, neuronal division and migration \cite{Norden:09}, immune response \cite{Cahalan:08} or the development of cancer \cite{Ianzini:09}. Based on phase-contrast, confocal or two-photon microscopy, such live cell imaging protocols are now commonly established and corresponding equipment is commonly available. This has triggered the need for computational methods to quantitatively analyze time-lapse microscopy data. In this context, identifying individual cells and tracking their identities over time is one of the basic ingredients for computational analysis. Hence, cell tracking algorithms have attracted considerable attention in recent years \cite{Meijering:06,Miura:05}. Here, we introduce a novel algorithm for cell tracking that allows tracking cells, in particular for zebra-fish microglia cells, in 3D two-photon image sequences over time.

The majority of cell tracking algorithms, as surveyed by Meijering \cite{Meijering:06} or Miura \cite{Miura:05}, deals with cell tracking in 2D over time. Methods range from linking cells identified in individual frames using different segmentation approaches to active-contour \cite{dufour2005segmenting, sacan2008celltrack, Shen:06} or level-set algorithms \cite{Dzyubachyk:08, Li:08, Mukherjee:04, Nath:06}. The challenges imposed by the nature of the images to be analyzed lie in phenomena such as cell divisions \cite{AlKofahi:06, Li:08}, cells entering or leaving the displayed area, or a large number of cells that needs to be tracked simultaneously. In addition, cell tracking is often complicated by background inhomogeneity, for instance due to uneven illumination \cite{Leong:03}, and cells touching each other. While these issues have been addressed extensively for tracking cells in 2D, surprisingly few approaches have addressed cell tracking in 3D. Besides naive thresholding approaches, there are only few advanced approaches, such as the active-contour based method proposed by Dufour \cite{dufour2005segmenting}. Recently, several authors \cite{Jaensch:10, Kerekes:09} proposed reliable methods for tracking centrosomes in Caenorhabditise legans embryos. Yet, these approaches are tailored toward tracking small, bright and circular objects which e.g. resemble a Gaussian spot of a specific size. Such assumptions, however, are not satisfied by the complex and highly variable shapes of microglia under consideration here. Cell tracking is also relevant in the context of tracking cell populations \emph{in vitro}, which has attracted considerable attention recently \cite{House:09, Ong:10, Padfield:09}.
\subsection{Motivation and main contribution}
\textbf{Motivation:} Cell tracking is an important method to quantitatively analyze time-lapse microscopy data. While numerous methods and tools exist for tracking cells in 2D time-lapse images, only few and very application-specific tracking tools are available for 3D time-lapse images, which is of high relevance in immunoimaging, in particular for studying the motility of microglia \emph{in vivo}. In this thesis, we aim to develop a cell tracking method that adapts most bioimage data.
\\
\textbf{Contribution:} We introduce a novel algorithm for tracking cells in 3D time-lapse microscopy data, based on computing cosegmentations between component trees representing individual time frames using the so-called tree-assignments. For the first time, our method allows to track microglia in three dimensional confocal time-lapse microscopy images. We also evaluate our method on synthetically generated data, demonstrating that our algorithm is robust even in the presence of different types of inhomogeneous background noise.

\section{Neuron Tracing}
\subsection{Background}
3D reconstruction of complex neuron morphology from light-microscopic images is an important technique for computational neuroscience. A number of studies \cite{al2002rapid, al2003median, abdul2005automatic, cai2008using, dima2002automatic, evers2005progress, meijering2004design, narro2007neuronmetrics, rodriguez2008automated, schmitt2004new, sun2009fast, vasilkoski2009detection, wearne2005new, weaver2004automated, xiong2006automated, yuan2009mdl, zhang2004tracking, zhang2007novel,zhang2007automated} have been focused to develop semi- or fully automatic neuron tracing methods. In recent years, it has received special attention, such as in the DIADEM competition \cite{brown2011diadem, gillette2011diademchallenge} that involved ~100 teams worldwide and many related studies \cite{al2002rapid, choromanska2012automatic, cohen1994automated, donohue2011automated, lu2009semi, meijering2004design, meijering2010neuron, narayanaswamy20113, narro2007neuronmetrics, peng2011automatic, peng2010v3d, peng2010automatic, vallotton2007automated, wang2011broadly, xiong2006automated, zhang2007novel, zhang2007automated, zhao2011automated}. However, despite a number of developed algorithms of neuron reconstruction (also called ‘neuron tracing’), it remains a significant problem on how to trace neurons in a robust and precise way from real 3D microscopic images.

\subsection{Motivation and main contribution}\label{sec:contrib}
\textbf{Motivation:} Tracing of neuron morphology is an essential technique in computational neuroscience. However, despite a number of existing methods, few open-source techniques are completely or sufficiently automated and at the same time are able to generate robust results for real 3D microscopy images semi- or fully automaticly. 
\\
\textbf{Contribution:} We developed all-path-pruning 2.0 (APP2) for 3D neuron tracing. The most important idea is to prune an initial reconstruction tree of a neuron’s morphology using a long-segment-first hierarchical procedure instead of the original termini-first-search process in APP. To further enhance the robustness of APP2, we compute the distance transform of all image voxels directly for a gray-scale image, without the need to binarize the image before invoking the conventional distance transform. We also design a fast-marching algorithm-based method to compute the initial reconstruction trees without pre-com- puting a large graph. This method allows us to trace large images. We bench-tested APP2 on $\sim 700$ 3D microscopic images and found that APP2 can generate more satisfactory results in most cases than sev- eral previous methods. 

The fast-marching algorithm also helps us to build an elegent way to draw 3D curves, namely human-guided neuron tracing. It is very robust and can get the exact 3D curve desired by finding the shortest path between view rays. 

\section{Thesis organization}
The rest of the thesis is organized as follows. Part 2 describes the cell tracking methods (chapter \ref{chpt:coseg}) and the related image segmentation (chapter \ref{chpt:imgseg}), component tree construction (chapter \ref{chpt:cptree}) and tree assignment methods (chapter \ref{chpt:treeassign}). Part 3 describes the human-guided neuron tracing method (chapter \ref{chpt:fm}) by using fast marching algorithm, and automatic neuron tracing method (chapter \ref{chpt:auto-nt}) by using grayscale distance transform, distance tree construction and hierarchical pruning method.

